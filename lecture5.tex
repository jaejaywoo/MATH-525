%% LyX 2.2.1 created this file.  For more info, see http://www.lyx.org/.
%% Do not edit unless you really know what you are doing.
\documentclass[english,12pt]{article}
\usepackage[T1]{fontenc}
\usepackage[latin9]{inputenc}
\usepackage{amsmath}
\usepackage{amsthm}
\usepackage{amssymb}

\makeatletter
%%%%%%%%%%%%%%%%%%%%%%%%%%%%%% Textclass specific LaTeX commands.
\theoremstyle{plain}
\newtheorem{thm}{\protect\theoremname}[section]
\theoremstyle{definition}
\newtheorem{defn}[thm]{\protect\definitionname}
\theoremstyle{plain}
\newtheorem{prop}[thm]{\protect\propositionname}
\ifx\proof\undefined
\newenvironment{proof}[1][\protect\proofname]{\par
\normalfont\topsep6\p@\@plus6\p@\relax
\trivlist
\itemindent\parindent
\item[\hskip\labelsep\scshape #1]\ignorespaces
}{%
\endtrivlist\@endpefalse
}
\providecommand{\proofname}{Proof}
\fi
\theoremstyle{definition}
\newtheorem{example}[thm]{\protect\examplename}
\theoremstyle{remark}
\newtheorem{rem}[thm]{\protect\remarkname}
\theoremstyle{definition}
\newtheorem{xca}[thm]{\protect\exercisename}

%%%%%%%%%%%%%%%%%%%%%%%%%%%%%% User specified LaTeX commands.
\usepackage[margin=1in]{geometry}

\makeatother

\usepackage{babel}
\providecommand{\definitionname}{Definition}
\providecommand{\examplename}{Example}
\providecommand{\exercisename}{Exercise}
\providecommand{\propositionname}{Proposition}
\providecommand{\remarkname}{Remark}
\providecommand{\theoremname}{Theorem}

\begin{document}

\title{Math 525: Lecture 5}

\date{January 18, 2018}
\maketitle

\section{Series (review)}
\begin{defn}
A sequence $(a_{n})_{n}\subset\mathbb{R}$ \emph{converges} to a point
$L\in\mathbb{R}$ (written $a_{n}\rightarrow L$ or $\lim_{n\rightarrow\infty}a_{n}=L$)
if for each $\epsilon>0$, we can find $N$ such that $|a_{n}-L|<\epsilon$
for all $n\geq N$. If the sequence does not converge to any point
in $\mathbb{R}$, we say it \emph{diverges}.
\end{defn}
In our case, we always use ``converge'' to mean ``converges to
a point in $\mathbb{R}$''. Depending on the context, sometimes people
will be talking about convergence in other spaces (e.g., if $a_{n}\rightarrow\infty$,
one might say the sequence converges to a point in $\overline{\mathbb{R}}=\mathbb{R}\cup\{-\infty,+\infty\}$:
this is a perfectly valid use of the terminology).

\begin{defn}
Let $(a_{n})_{n=1}^{\infty}\subset\mathbb{R}$ be a sequence. Then...
\begin{itemize}
\item We say the series $\sum_{n=1}^{\infty}a_{n}$ \emph{converges} (written
$\sum_{n=1}^{\infty}a_{n}<\infty$) if the sequence of partial sums
$(s_{N})_{N=1}^{\infty}$ defined by $s_{N}=\sum_{n=1}^{N}a_{n}$
converges. Otherwise, we say it \emph{diverges}. In the convergent
case, we define $\sum_{n=1}^{\infty}a_{n}=\lim_{N\rightarrow\infty}s_{N}$.
\item The series $\sum_{n=1}^{\infty}a_{n}$ \emph{converges absolutely}
if the series $\sum_{n=1}^{\infty}|a_{n}|$ converges.
\end{itemize}
\end{defn}
\begin{prop}
If a series converges \textbf{absolutely} to $L\geq0$, the series
converges to a number in $[-L,L]$.
\end{prop}
\begin{proof}
First, note that if $\sum_{n}|a_{n}|\rightarrow L$, then
\[
-L\leftarrow-\sum_{n=1}^{N}\left|a_{n}\right|\leq\sum_{n=1}^{N}a_{n}\leq\sum_{n=1}^{N}\left|a_{n}\right|\rightarrow L.\qedhere
\]
The remainder of the proof requires knowledge of Cauchy sequences
(if you are not familiar with them, you can safely skip this proof).
Suppose $\sum_{n=1}^{\infty}a_{n}$ converges absolutely. Define $s_{N}=\sum_{n=1}^{N}a_{n}$
and $S_{N}=\sum_{n=1}^{N}|a_{n}|$. Then, for $N>M$,
\[
\left|s_{N}-s_{M}\right|=\left|a_{N}+a_{N+1}+\cdots+a_{M+1}\right|\leq\left|a_{N}\right|+\left|a_{N+1}\right|+\cdots+\left|a_{M+1}\right|=\left|S_{N}-S_{M}\right|.
\]
Since $(S_{N})_{N}$ is convergent, it is a Cauchy sequence. From
the above, we see that $(s_{N})_{N}$ is a Cauchy sequence and hence
convergent.
\end{proof}
Rearranging the terms in a series may change its value. However, in
some cases, we can safely rearrange the terms in a series.
\begin{prop}
If a series is made up only of positive terms, it can be rearranged
without changing its sum.
\end{prop}

\section{Expectation (discrete case)}

In this lecture, we define expectations for discrete random variables.
Handling the discrete case separately serves the purpose of building
our intuition of expectations before we handle the more difficult
case of non-discrete random variables.

Recall that for a discrete random variable $X$, we can find a countable
set $\{x_{n}\}_{n}\subset\mathbb{R}$ such that
\[
\sum_{n}\mathbb{P}(\left\{ X=x_{n}\right\} )=1.
\]
Note that this does not necessarily imply that the range of $X$ is
$\{x_{n}\}_{n}$ (remember, random variables are functions from $\Omega$
to $\mathbb{R}$). However, we can define a new random variable, call
it $Y$, as follows: 
\[
Y(\omega)=\sum_{n}x_{n}I_{\{X=x_{n}\}}(\omega).
\]
Note that
\[
\mathbb{P}(\left\{ Y=x_{n}\right\} )=\mathbb{P}(\left\{ I_{\{X=x_{n}\}}=1\right\} )=\mathbb{P}(\left\{ X=x_{n}\right\} ),
\]
and hence the random variables $X$ and $Y$ are, for all intents
and purposes, identical. Therefore, for the remainder, we will always
assume\textemdash without loss of generality\textendash that a discrete
random variable $X$ has the form
\[
X(\omega)=\sum_{n}x_{n}I_{\Lambda_{n}}(\omega)
\]
for some partition $\Lambda_{1},\Lambda_{2},\ldots$ of the sample
space (i.e., $\Omega=\Lambda_{1}\cup\Lambda_{2}\cup\cdots$ and $\Lambda_{i}\cap\Lambda_{j}=\emptyset$
whenever $i\neq j$).
\begin{defn}
A discrete random variable $X$ is \emph{integrable} if
\[
\sum_{n}\left|x_{n}\right|\mathbb{P}(\Lambda_{n})<\infty.
\]
\end{defn}
%
\begin{defn}
The \emph{expectation} of an integrable discrete random variable $X$
is
\[
\mathbb{E}X=\sum_{n}x_{n}\mathbb{P}(\Lambda_{n}).
\]
\end{defn}
\begin{example}
Toss a coin $N\geq1$ times. Let $X_{N}$ be the number of heads.
If the probability of heads is $p$, the expectation of $X_{N}$ is
\[
\mathbb{E}X_{N}=\sum_{n=0}^{N}n\mathbb{P}(\left\{ X_{n}=n\right\} )=\sum_{n=1}^{N}n\binom{N}{n}p^{n}(1-p)^{N-n}=pN.
\]
That is, you are expected to see $pN$ heads ``on average''. If
the coin is fair, for example, $pN=N/2$ (half of the tosses should,
on average, be heads).
\end{example}
Note, in particular, that we only define the expectation of random
variables that are integrable, and integrability has to do with absolute
convergence.
\begin{rem}
We have, so far, ignored a technical issue. Earlier, we characterized
a random variable in terms of the sets $\Lambda_{1},\Lambda_{2},\ldots$
However, the choice of these sets is not unique. For example, the
constant random variable
\[
X(\omega)=1
\]
can be written in many ways. Two possibilities are $X(\omega)=1I_{\{\Omega\}}(\omega)$
and $X(\omega)=1I_{\{\Lambda\}}(\omega)+1I_{\{\Lambda^{c}\}}(\omega)$
where $\Lambda$ is any subset of the sample space $\Omega$. Since
the definition of expectation depends on a particular choice of the
sets $\Lambda_{1},\Lambda_{2},\ldots$, it is not clear that the expectation
will remain the same if we change our choice of $\Lambda_{1},\Lambda_{2},\ldots$
This technicality is handled on page 55 of Walsh, John B. \emph{Knowing
the odds: an introduction to probability}. Vol. 139. American Mathematical
Soc., 2012.
\end{rem}
\begin{example}
Let $X$ be a nonnegative integer-valued random variable (i.e., $\sum_{n\geq0}\mathbb{P}(\{X=n\})=1$).
We assume $X$ has the form
\[
X(\omega)=\sum_{n\geq0}nI_{\{X=n\}}(\omega).
\]
If $X$ is integrable,
\begin{multline*}
\mathbb{E}\left[X\right]=\sum_{n\geq0}n\mathbb{P}(\left\{ X=n\right\} )=0\mathbb{P}(\left\{ X=0\right\} )+1\mathbb{P}(\left\{ X=1\right\} )+2\mathbb{P}(\left\{ X=2\right\} )+\cdots\\
=\left(\mathbb{P}(\left\{ X=1\right\} )+\mathbb{P}(\left\{ X=2\right\} )+\cdots\right)+\left(\mathbb{P}(\left\{ X=2\right\} )+\mathbb{P}(\left\{ X=3\right\} )+\cdots\right)=\sum_{n\geq1}\mathbb{P}(\left\{ X\geq n\right\} ).
\end{multline*}
\end{example}
\begin{prop}
Let $X$ and $Y$ be discrete random variables and $a,b\in\mathbb{R}$.
Then,
\begin{enumerate}
\item If $X$ and $Y$ are integrable, so is $aX+bY$ and
\[
\mathbb{E}\left[aX+bY\right]=a\mathbb{E}X+b\mathbb{E}Y.
\]
\item If $|X|\leq|Y|$ and $Y$ is integrable, then $X$ is integrable.
\item If $X$ and $Y$ are integrable and $X\leq Y$, then $\mathbb{E}X\leq\mathbb{E}Y$.
\item If $X$ is integrable, $\mathbb{E}X\leq\mathbb{E}|X|$.
\end{enumerate}
\end{prop}
\begin{proof}
Recall that we can partition $\Omega$ into events $\Lambda_{1}^{X},\Lambda_{2}^{X},\ldots$
on which $X$ is constant. We can do the same for $Y$, obtaining
$\Lambda_{1}^{Y},\Lambda_{2}^{Y},\ldots$ This allows us to define
$\Lambda_{ij}=\Lambda_{i}^{X}\cap\Lambda_{j}^{Y}$, on which \textbf{both}
$X$ and $Y$ are constant. Since $(\Lambda_{ij})_{i,j}$ is a countable
sequence, let's relabel it $(\Lambda_{n})_{n}$ and take $X=x_{n}$
and $Y=y_{n}$ on $\Lambda_{n}$.
\begin{enumerate}
\item Suppose $X$ and $Y$ are integrable. Then,
\begin{align*}
\sum\left|ax_{n}+by_{n}\right|\mathbb{P}(\Lambda_{n}) & \leq\left|a\right|\sum\left|x_{n}\right|\mathbb{P}(\Lambda_{n})+\left|b\right|\sum\left|y_{n}\right|\mathbb{P}(\Lambda_{n})\\
 & =\left|a\right|\sum\left|x_{n}\right|\mathbb{P}(\Lambda_{n}^{X})+\left|b\right|\sum\left|y_{n}\right|\mathbb{P}(\Lambda_{n}^{Y})\\
 & =\left|a\right|\mathbb{E}\left[\left|X\right|\right]+\left|b\right|\mathbb{E}\left[\left|Y\right|\right]
\end{align*}
and hence $aX+bY$ is integrable. Repeating almost the exact same
computation as above without the absolute value signs yields the desired
result.
\item This follows from $\sum|x_{n}|\mathbb{P}(\Lambda_{n})\leq\sum|y_{n}|\mathbb{P}(\Lambda_{n})$.
\item Exercise.
\item Take $Y=|X|$ in (3).\qedhere
\end{enumerate}
\end{proof}
Most importantly, the above proposition tells us that the expectation
is a linear function. That is, let $\mathcal{X}$ be the set of all
random variables. Define $T\colon\mathcal{X}\rightarrow\mathbb{R}$
as the mapping from a random variable to its expectation:
\[
T(X)=\mathbb{E}X.
\]
Then, $T$ is linear function (i.e., $T(aX+bY)=aT(X)+bT(Y)$).

As an example of expectations, we introduce now \emph{probability
generating function} of a discrete random variable. We point out that
our treatment is a bit cavalier for the time being, but we will come
back to generating functions in a more principled manner. Before we
move to this example, let's give a simple definition:
\begin{defn}
The \emph{probability mass function} (PMF) of a discrete random variable
$X$ is $p\colon\mathbb{R}\rightarrow[0,1]$ defined by
\[
p(x)=\begin{cases}
\mathbb{P}(\Lambda_{n}) & \text{if }x=x_{n}\\
0 & \text{otherwise}.
\end{cases}
\]
\end{defn}
\begin{example}
Let $X$ be a nonnegative integer-valued random variable. Define $G$,
the probability generating function of $X$, by
\[
G(t)=\mathbb{E}\left[t^{X}\right].
\]
Ignoring the integrability of $X$,
\[
G(t)=\sum_{n=0}^{\infty}p(n)t^{n}=p(0)+\sum_{n=1}^{\infty}p(n)t^{n}
\]
where $p$ is the probability mass function of $X$. Being a power
series, $G$ has a radius convergence $0\leq R\leq\infty$ which characterizes
which values of $t$ it converges for (i.e., converges for $|t|<R$
and diverges for $t>|R|$). Since
\[
G(1)=\sum_{n=0}^{\infty}p(n)1^{n}=\sum_{n=0}^{\infty}p(n)=1,
\]
we know that the radius of convergence must be at least one (i.e.,
$R\geq1$). Furthermore,
\[
G(0)=p(0)+\sum_{n=1}^{\infty}p(n)0^{n}=p(0).
\]
Now, if we take derivatives of $G$ (for values of $t$ inside the
radius of convergence), we get
\begin{align*}
G^{\prime}(t) & =\sum_{n=1}^{\infty}np(n)t^{n-1}\\
G^{\prime\prime}(t) & =\sum_{n=2}^{\infty}n\left(n-1\right)p(n)t^{n-2}\\
 & \vdots\\
G^{(k)}(t) & =\sum_{n=k}^{\infty}n\left(n-1\right)\cdots\left(n-k+1\right)p(n)t^{n-k}.
\end{align*}
We conclude that $p(n)=\frac{1}{n!}G^{(n)}(0)$ for $n=1,2,\ldots$
\end{example}
In the previous example, we wrote $\mathbb{E}[t^{X}]$ even though
$t^{X}$ was not necessarily integrable for arbitrary $t$. We will
often perform this abuse of notation by writing $\mathbb{E}Y$ for
any random variable $Y$ with the implicit understanding that the
$\mathbb{E}Y$ is only well-defined when $Y$ is integrable.
\begin{prop}
Let $f\colon\mathbb{R}\rightarrow\mathbb{R}$ and $X$ be a discrete
random variable with probability mass function $p$ and support $\{x_{n}\}_{n}$.
Then, $Y=f\circ X$ is integrable if and only if $\sum_{n}|f(x_{n})|p(x_{n})<\infty$,
in which case
\[
\mathbb{\mathbb{E}}\left[f(X)\right]=\sum_{n}f(x_{n})p(x_{n}).
\]
\end{prop}

\section{Variance}
\begin{defn}
Let $X$ be a discrete random variable. Its \emph{variance} is defined
as
\[
\operatorname{Var}X=\mathbb{E}\left[\left(X-\mathbb{E}X\right)^{2}\right]
\]
(there is an implicit assumption about integrability in the definition
of variance). Its \emph{standard deviation} is $\sqrt{\operatorname{Var}X}$.
\end{defn}
Once again interpreting $\mathbb{E}$ as an average, it is clear from
the definition that variance is a measure of how far the random variable
is from the expectation ``on average''. Note also that
\begin{align*}
\operatorname{Var}X & =\mathbb{E}\left[\left(X-\mathbb{E}X\right)^{2}\right]\\
 & =\mathbb{E}\left[X^{2}-2X\mathbb{E}X+\left(\mathbb{E}X\right)^{2}\right]\\
 & =\mathbb{E}\left[X^{2}\right]-2\mathbb{E}X\mathbb{E}X+\left(\mathbb{E}X\right)^{2}\\
 & =\mathbb{E}\left[X^{2}\right]-2\left(\mathbb{E}X\right)^{2}+\left(\mathbb{E}X\right)^{2}\\
 & =\mathbb{E}\left[X^{2}\right]-\left(\mathbb{E}X\right)^{2},
\end{align*}
which gives us a useful formula for the variance of a random variable.
We will see next class that $\mathbb{E}X^{2}$ is referred to as the\emph{
second }\textbf{\emph{raw}}\emph{ moment}, and another name for the
variance is the \emph{second }\textbf{\emph{central}}\emph{ moment}.
\begin{example}
Toss a coin $N\geq1$ times. Let $X_{N}$ be the number of heads.
Remember, we computed $\mathbb{E}X_{N}=pN$. Therefore, to get $\operatorname{Var}X_{N}$,
it is sufficient to compute $\mathbb{E}[X_{N}^{2}]$:
\[
\mathbb{E}\left[X_{N}^{2}\right]=\sum_{n=0}^{N}n^{2}\mathbb{P}(\left\{ X_{N}=n\right\} )=\sum_{n=1}^{N}n^{2}\binom{N}{n}p^{n}(1-p)^{N-n}=p\left(\left(N-1\right)Np+N\right).
\]
\end{example}
\begin{xca}
Let $X$ be a random variable with finite variance. Then
\[
\operatorname{Var}(aX+b)=a^{2}\operatorname{Var}(X).
\]
\end{xca}
%
\begin{xca}
Let $X$ be a discrete random variable. Show that if $X^{2}$ is integrable,
then $X$ is integrable (i.e., it is not possible to have a random
variable with finite variance but infinite expectation).
\end{xca}
\begin{rem}
For those familiar with measure theory, the above is an immediate
consequence of the deeper fact that for a finite measure space, $\mathbb{L}^{q}\subset\mathbb{L}^{p}$
for $1\leq p\leq q\leq\infty$.
\end{rem}

\end{document}
