%% LyX 2.2.3 created this file.  For more info, see http://www.lyx.org/.
%% Do not edit unless you really know what you are doing.
\documentclass[english,12pt]{article}
\usepackage[T1]{fontenc}
\usepackage[latin9]{inputenc}
\usepackage{color}
\usepackage{babel}
\usepackage{mathtools}
\usepackage{amsmath}
\usepackage{amssymb}
\usepackage[unicode=true,pdfusetitle,
 bookmarks=true,bookmarksnumbered=false,bookmarksopen=false,
 breaklinks=false,pdfborder={0 0 1},backref=false,colorlinks=true]
 {hyperref}
\usepackage{breakurl}

\makeatletter
%%%%%%%%%%%%%%%%%%%%%%%%%%%%%% User specified LaTeX commands.
\usepackage[margin=1in]{geometry}

\makeatother

\begin{document}

\title{Math 525: Assignment 5 Solutions}

\date{\date{}}
\maketitle
\begin{enumerate}
\item This is just a consequence of Cauchy-Schwarz:
\[
\mathbb{E}\left[XYZ\right]\leq\sqrt{\mathbb{E}\left[(XY)^{2}\right]\mathbb{E}\left[Z^{2}\right]}=\sqrt{\mathbb{E}\left[X^{2}\right]\mathbb{E}\left[Y^{2}\right]\mathbb{E}\left[Z^{2}\right]}.
\]
We have used independence in the last equality.
\item Since $x\mapsto e^{\theta x}$ is convex, this is just a consequence
of Jensen's inequality:
\[
e^{\theta\mathbb{E}X}\leq\mathbb{E}\left[e^{\theta X}\right]=M(\theta).
\]
\item Let $\epsilon>0$ and
\[
\Lambda_{n}^{\epsilon}=\left\{ \left|X_{n}\right|^{p}\geq\epsilon\right\} .
\]
By Chebyshev's inequality,
\[
\mathbb{P}(\Lambda_{n})\leq\frac{1}{\epsilon^{p}}\mathbb{E}\left[\left|X_{n}\right|^{p}\right]\leq\frac{1}{\epsilon^{p}}f(n).
\]
Therefore, $\sum_{n}\mathbb{P}(\Lambda_{n})<\infty$ and hence by
Borel-Cantelli,
\[
\mathbb{P}(\limsup_{n}\Lambda_{n}^{\epsilon})=0.
\]
Now, consider
\[
\Lambda=\bigcap_{\substack{\epsilon>0\\
\epsilon\in\mathbb{Q}
}
}\left(\limsup_{n}\Lambda_{n}^{\epsilon}\right).
\]
If $\omega\notin\Lambda$, $X_{n}(\omega)\rightarrow0$, as desired.
\end{enumerate}

\end{document}
