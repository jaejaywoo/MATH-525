%% LyX 2.2.3 created this file.  For more info, see http://www.lyx.org/.
%% Do not edit unless you really know what you are doing.
\documentclass[english,12pt,addpoints]{exam}
\usepackage[T1]{fontenc}
\usepackage[latin9]{inputenc}
\setcounter{secnumdepth}{2}
\setcounter{tocdepth}{2}
\usepackage{xcolor}
\usepackage{fancybox}
\usepackage{calc}
\usepackage{units}
\usepackage{amsmath}
\usepackage{amssymb}
\PassOptionsToPackage{normalem}{ulem}
\usepackage{ulem}

\makeatletter
%%%%%%%%%%%%%%%%%%%%%%%%%%%%%% Textclass specific LaTeX commands.
%% Map the lyx-friendly SmartQuestion style to exam environments
%% See http://tex.stackexchange.com/a/319996/17868
\newcounter{mydepth}%
\newcommand{\myenvironmentname}{%
\ifnum\value{mydepth}=0notinquestion\fi%
\ifnum\value{mydepth}=1question\fi%
\ifnum\value{mydepth}=2part\fi%
\ifnum\value{mydepth}=3subpart\fi%
\ifnum\value{mydepth}=4subsubpart\fi%
\ifnum\value{mydepth}>4undefinedquestion\fi%
}%
\newcommand{\myquestion}{\csname \myenvironmentname\endcsname}%
\newenvironment{myquestions}%
  {\stepcounter{mydepth}\begin{\myenvironmentname s}}%
  {\end{\myenvironmentname s}\addtocounter{mydepth}{-1}}%

% code to rewrite \mychoice or \mychoice[] to \choice or \CorrectChoice respectively
% see http://tex.stackexchange.com/a/319470/17868
\def\mychoice{\@ifnextchar[{\@with}{\@without}}
\def\@with[#1]{\CorrectChoice}
\def\@without{\choice}


\makeatother

\usepackage{babel}
\begin{document}

\footer{}{}{Thursday, March 20, 2018}{}

\title{Midterm 2}

\date{\date{}}

\author{Math 525: Probability}

\maketitle
\thispagestyle{headandfoot} % hack to work around \maketitle clobbering the first page's headers/footers
\begin{center}
\noindent\begin{minipage}[t]{1\textwidth}%
Last name, first name:\enskip{}\hrulefill{}~%
\end{minipage}\vspace{0.2in}
\noindent\begin{minipage}[t]{1\textwidth}%
Section number:\enskip{}\hrulefill{}~%
\end{minipage}\vspace{0.2in}
\noindent\begin{minipage}[t]{1\textwidth}%
User ID:\enskip{}\hrulefill{}~%
\end{minipage}\vfill{}
\noindent\fcolorbox{black}{lightgray}{\parbox[t]{1\columnwidth - 2\fboxsep - 2\fboxrule}{%
\begin{center}
\gradetable[h]{}
\par\end{center}%
}}\vfill{}
\doublebox{\begin{minipage}[t]{5.5in}%
\begin{center}
Answer the questions in the spaces provided on the question sheets.
If you run out of room for an answer, continue on the back of the
page.
\par\end{center}%
\end{minipage}}
\par\end{center}

\noindent \begin{flushleft}
\qformat{\textbf{Question \thequestion}\quad (\totalpoints\ \points)\hfill}
\pointsinrightmargin
\printanswers
\par\end{flushleft}

\noindent \newpage{}\vspace{0.1in}
\begin{myquestions}
\myquestion[35] Consider a person walking along a line. Their position at time $n$
is denoted $X_{n}$, and they start at position zero (i.e., $X_{0}=0$).
At any point in time, the person can either take a step forward ($X_{n+1}=X_{n}+1$)
or a step backwards ($X_{n+1}=X_{n}-1$). Let $N$ be a positive integer
and $0<p<1$. The person walking follows the rules
\begin{align*}
\mathbb{P}(X_{n+1}=i+1\mid X_{n}=i) & =p, & \text{if }-N<i<N\\
\mathbb{P}(X_{n+1}=i-1\mid X_{n}=i) & =1-p, & \text{if }-N<i<N
\end{align*}
and
\[
\mathbb{P}(X_{n+1}=\pm(N-1)\mid X_{n}=\pm N)=1
\]
\begin{myquestions}
\myquestion Write down the transition matrix $P$ for $N=2$ and $p=\nicefrac{1}{3}$.
\myquestion Recalling that $X_{0}=0$, give an expression for $\mathbb{P}(X_{10}=0)$
using the matrix $P$.\\
Just write down an expression: do not compute anything!
\end{myquestions}

\newpage{}
\begin{solution}
XXX
\end{solution}

\myquestion[35] Recall that the characteristic function of a normal random variable
$X_{n}\sim\mathcal{N}(0,n)$ with mean zero and variance $n$ is
\[
\phi_{n}(t)=e^{-nt^{2}/2}.
\]
\begin{myquestions}
\myquestion What is $\phi(t)=\lim_{n\rightarrow\infty}\phi_{n}(t)$?
\myquestion Can we conclude, \uline{using L�vy's continuity theorem}, that
$X_{n}$ converges in distribution to a random variable $X$ with
characteristic function $\phi$? Why or why not?
\end{myquestions}
\textbf{L�vy's continuity theorem}: Let $X_{n}$ be a random variable
with distribution functions $F_{n}$ and characteristic function $\phi_{n}$.
If $\lim_{n\rightarrow\infty}\phi_{n}(t)=\phi(t)$ for some function
$\phi$ which is continuous at the origin, then there exists a distribution
function $F$ such that $F_{n}\Rightarrow F$ and $\phi$ is the characteristic
function of $F$.

\newpage{}
\begin{solution}
XXX
\end{solution}

\myquestion[30] We would like to compute
\[
\mathbb{E}\left[I_{A}(X)\right]
\]
where $A$ is some (Borel) subset of the real line and $X$ is a random
variable. One way to do so involves generating independent samples
$X_{1},\ldots,X_{n}$ (each having the same distribution as $X$)
and making the approximation 
\[
\mathbb{E}\left[I_{A}(X)\right]\approx\frac{S_{n}}{n}\qquad\text{where}\qquad S_{n}=I_{A}(X_{1})+\cdots+I_{A}(X_{n}).
\]
The central limit theorem tells us
\begin{equation}
\sqrt{n}\left(\frac{S_{n}}{n}-\mathbb{E}\left[I_{A}(X)\right]\right)\xrightarrow{\mathcal{D}}\mathcal{N}(0,\sigma^{2}).\tag{*}\label{eq:clt}
\end{equation}
\begin{myquestions}
\myquestion What is the exact value of $\sigma^{2}$ in the above? Simplify as
much as possible.
\myquestion In your own words, what is the significance of (\ref{eq:clt})?
\end{myquestions}
\end{myquestions}

\end{document}
