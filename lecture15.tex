%% LyX 2.2.3 created this file.  For more info, see http://www.lyx.org/.
%% Do not edit unless you really know what you are doing.
\documentclass[english,12pt]{article}
\usepackage[T1]{fontenc}
\usepackage[latin9]{inputenc}
\usepackage{mathtools}
\usepackage{amsmath}
\usepackage{amsthm}
\usepackage{amssymb}

\makeatletter
%%%%%%%%%%%%%%%%%%%%%%%%%%%%%% Textclass specific LaTeX commands.
\theoremstyle{plain}
\newtheorem{thm}{\protect\theoremname}[section]
\theoremstyle{definition}
\newtheorem{defn}[thm]{\protect\definitionname}
\theoremstyle{plain}
\newtheorem{prop}[thm]{\protect\propositionname}
\ifx\proof\undefined
\newenvironment{proof}[1][\protect\proofname]{\par
\normalfont\topsep6\p@\@plus6\p@\relax
\trivlist
\itemindent\parindent
\item[\hskip\labelsep\scshape #1]\ignorespaces
}{%
\endtrivlist\@endpefalse
}
\providecommand{\proofname}{Proof}
\fi
\theoremstyle{definition}
\newtheorem{xca}[thm]{\protect\exercisename}
\theoremstyle{definition}
\newtheorem{example}[thm]{\protect\examplename}

%%%%%%%%%%%%%%%%%%%%%%%%%%%%%% User specified LaTeX commands.
\usepackage[margin=1in]{geometry}

\makeatother

\usepackage{babel}
\providecommand{\definitionname}{Definition}
\providecommand{\examplename}{Example}
\providecommand{\exercisename}{Exercise}
\providecommand{\propositionname}{Proposition}
\providecommand{\theoremname}{Theorem}

\begin{document}

\title{Math 525: Lecture 15}

\date{March 6, 2018}

\maketitle
Today, we finally prove the central limit theorem (CLT). Recall that
for a ``sufficiently nice'' sequence of i.i.d. random variables
$(X_{n})_{n}$, the law of large numbers told us
\[
\frac{S_{n}}{n}\equiv\frac{X_{1}+\cdots+X_{n}}{n}\rightarrow\mathbb{E}X_{1}\text{ a.s.}
\]
The CLT will tell us about the distribution of $S_{n}/n$. Namely,
\begin{equation}
\sqrt{n}\left(\frac{S_{n}}{n}-\mathbb{E}X_{1}\right)\xrightarrow{\mathcal{D}}Y\label{eq:clt}
\end{equation}
where $Y\sim\mathcal{N}(0,\operatorname{Var}(X_{1}))$. Before we
give the CLT, let's review normal random variables.

\section{Normal random variables}
\begin{defn}
We say $X$ is a \emph{normal random variable} with mean $\mu$ and
variance $\sigma^{2}$ if its probability density is
\[
f(x)=\frac{1}{\sqrt{2\pi\sigma^{2}}}e^{-\frac{(x-\mu)^{2}}{2\sigma^{2}}}.
\]
In this case, we write $X\sim\mathcal{N}(\mu,\sigma^{2})$.
\end{defn}
\begin{prop}
The characteristic function of $X\sim\mathcal{N}(0,1)$ random variable
is
\[
\phi_{X}(t)=e^{-t^{2}/2}.
\]
\end{prop}
\begin{proof}
Note that
\begin{align*}
\phi_{X}(t)=\mathbb{E}\left[e^{itX}\right] & =\frac{1}{\sqrt{2\pi}}\int_{-\infty}^{\infty}e^{-\frac{x^{2}}{2}}e^{itx}dx\\
 & =\frac{1}{\sqrt{2\pi}}\left(\int_{-\infty}^{0}e^{-\frac{x^{2}}{2}}e^{itx}dx+\int_{0}^{\infty}e^{-\frac{x^{2}}{2}}e^{itx}dx\right)\\
 & =\frac{1}{\sqrt{2\pi}}\left(\int_{-\infty}^{0}e^{-\frac{x^{2}}{2}}e^{itx}dx+\int_{0}^{\infty}e^{-\frac{x^{2}}{2}}e^{-itx}dx\right)\\
 & =\frac{2}{\sqrt{2\pi}}\int_{0}^{\infty}e^{-\frac{x^{2}}{2}}\cos(tx)dx.
\end{align*}
Now, take the derivative with respect to $t$ to get
\[
\phi_{X}^{\prime}(t)=-\frac{2}{\sqrt{2\pi}}\int_{0}^{\infty}e^{-\frac{x^{2}}{2}}x\sin(tx)dx.
\]
Integrate by parts to get
\begin{align*}
\phi_{X}^{\prime}(t) & =\frac{2}{\sqrt{2\pi}}\left(e^{-\frac{x^{2}}{2}}\sin(tx)\mid_{0}^{\infty}-t\int_{0}^{\infty}e^{-\frac{x^{2}}{2}}\cos(tx)dx\right)\\
 & =-\frac{2t}{\sqrt{2\pi}}\int_{0}^{\infty}e^{-\frac{x^{2}}{2}}\cos(tx)dx\\
 & =-t\phi_{X}(t).
\end{align*}
Note that
\[
\phi^{\prime}(t)=-t\phi(t)
\]
is an ordinary differential equation with
\[
\phi(t)=C_{0}\exp\left(-\frac{t^{2}}{2}\right)
\]
where $C_{0}$ is some constant. Since $\phi$ is a characteristic
function, we must have $\phi(0)=1$. This implies that $C_{0}=1$.
\end{proof}

\section{Classical CLT}

Just like with the law of large numbers, there are various versions
of the CLT. Here is the first version we encounter:
\begin{prop}
Let $(X_{j})_{j}$ be a sequence of i.i.d. random variables. Let $\mu=\mathbb{E}[X_{1}]$,
$\sigma^{2}=\operatorname{Var}(X_{1})$, and $S_{n}=X_{1}+\cdots+X_{n}$.
Then,
\[
\frac{\sqrt{n}}{\sigma}\left(\frac{S_{n}}{n}-\mu\right)\xrightarrow{\mathcal{D}}\mathcal{N}(0,1)
\]
(note that this is identical to (\ref{eq:clt}) if we multiply both
sides by $\sigma$).
\end{prop}
In establishing Levy's continuity theorem, we have actually done all
the hard work already to prove this fact.
\begin{proof}
First, note that
\[
\frac{\sqrt{n}}{\sigma}\left(\frac{S_{n}}{n}-\mu\right)=\frac{\left(X_{1}-\mu\right)+\cdots+\left(X_{n}-\mu\right)}{\sqrt{n}\sigma}=\frac{Y_{1}+\cdots+Y_{n}}{\sqrt{n}}.
\]
where
\[
Y_{n}=\frac{X_{n}-\mu}{\sigma}.
\]
Note that
\begin{align*}
\mathbb{E}Y_{n} & =\frac{\mathbb{E}X_{n}-\mu}{\sigma}=0\\
\operatorname{Var}(Y_{n}) & =\operatorname{Var}\left(\frac{X_{n}-\mu}{\sigma}\right)=1.
\end{align*}
For brevity, let $S_{n}^{\prime}=Y_{1}+\cdots+Y_{n}$.

The characteristic function of $Y_{n}$ is
\begin{align*}
\phi(t) & =\mathbb{E}[e^{itY_{n}}]\\
 & =\mathbb{E}\left[\sum_{k\geq0}\frac{\left(it\right)^{k}}{k!}Y_{n}^{k}\right]\\
 & =\mathbb{E}\left[Y^{0}\right]+\left(it\right)\mathbb{E}\left[Y^{1}\right]+\frac{\left(it\right)^{2}}{2}\mathbb{E}\left[Y^{2}\right]+\cdots\\
 & =1-\frac{t^{2}}{2}+\cdots
\end{align*}
More precisely,
\[
\phi(t)=1-\frac{t^{2}}{2}+h(t^{2})
\]
where $h$ denotes a function that satisfying $h(cx)/x\rightarrow0$
as $x\rightarrow0$ (corresponding to the higher order terms in the
Taylor expansion). Now, let
\[
S_{n}^{\prime}=\frac{Y_{1}+\cdots+Y_{n}}{\sqrt{n}}.
\]
Note that
\begin{align*}
\phi_{S_{n}^{\prime}}(t) & =\mathbb{E}\left[\exp\left(it\frac{Y_{1}+\cdots+Y_{n}}{\sqrt{n}}\right)\right]\\
 & =\mathbb{E}\left[\exp\left(it\frac{Y_{1}}{\sqrt{n}}\right)\cdots\exp\left(it\frac{Y_{n}}{\sqrt{n}}\right)\right]\\
 & =\mathbb{E}\left[\exp\left(it\frac{Y_{1}}{\sqrt{n}}\right)\right]\cdots\mathbb{E}\left[\exp\left(it\frac{Y_{n}}{\sqrt{n}}\right)\right]\\
 & =\phi\left(\frac{t}{\sqrt{n}}\right)\cdots\phi\left(\frac{t}{\sqrt{n}}\right)\\
 & =\left(\phi\left(\frac{t}{\sqrt{n}}\right)\right)^{n}\\
 & =\left(1-\frac{t^{2}}{2n}+o\left(\frac{t^{2}}{n}\right)\right)^{n}.\\
 & =\left(1+\frac{z_{n}}{n}\right)^{n}
\end{align*}
where we have defined
\[
z_{n}=-\frac{t^{2}}{2}+nh\left(\frac{t^{2}}{n}\right).
\]
Let $c=t^{2}$ and $x=1/n$. Then,
\[
nh\left(\frac{t^{2}}{n}\right)=\frac{h\left(cx\right)}{x}\rightarrow0\qquad\text{as}\qquad x\rightarrow0.
\]
Therefore,
\[
\phi_{S_{n}^{\prime}}(t)\rightarrow e^{t^{2}/2}.
\]
By L�vy's continuity theorem, $S_{n}^{\prime}$ converges to a standard
normal random variable.
\end{proof}
\begin{xca}
$n$ numbers are rounded to the nearest integer and then summed. Suppose
that the individual roundoff errors are uniformly distributed over
$[-0.5,0.5]$. What is the probability that the roundoff error exceeds
the exact sum by more than $x$?

Hint: let $Y_{j}$ be the $j$-th number and $[Y_{j}]$ be the result
from rounding. Then,
\[
X_{j}=Y_{j}-[Y_{j}]\sim U[-0.5,0.5].
\]
Therefore,
\[
\mathbb{E}X_{j}=0\qquad\text{and}\qquad\sigma^{2}=\operatorname{Var}(X_{j})=\mathbb{E}X_{j}^{2}=\int_{-1/2}^{1/2}x^{2}dx=\frac{1}{12}.
\]
Let $S_{n}=X_{1}+\cdots+X_{n}$. Then,
\begin{align*}
\mathbb{P}\left\{ \left|S_{n}\right|>x\right\}  & =1-\mathbb{P}\left\{ \left|S_{n}\right|\leq x\right\} \\
 & =1-\mathbb{P}\left\{ \frac{\left|S_{n}\right|}{\sqrt{n}\sigma}\leq\frac{x}{\sqrt{n}\sigma}\right\} \\
 & =1-\mathbb{P}\left\{ -\frac{x}{\sqrt{n}\sigma}\leq\frac{S_{n}}{\sqrt{n}\sigma}\leq\frac{x}{\sqrt{n}\sigma}\right\} \\
 & =1-\left(\mathbb{P}\left\{ \frac{S_{n}}{\sqrt{n}\sigma}\leq\frac{x}{\sqrt{n}\sigma}\right\} -\mathbb{P}\left\{ \frac{S_{n}}{\sqrt{n}\sigma}<-\frac{x}{\sqrt{n}\sigma}\right\} \right)\\
 & =1-\left(\mathbb{P}\left\{ \frac{S_{n}}{\sqrt{n}\sigma}\leq\frac{x}{\sqrt{n}\sigma}\right\} -\left(1-\mathbb{P}\left\{ \frac{S_{n}}{\sqrt{n}\sigma}\geq-\frac{x}{\sqrt{n}\sigma}\right\} \right)\right)\\
 & \approx1-\left(\mathbb{P}\left\{ \frac{S_{n}}{\sqrt{n}\sigma}\leq\frac{x}{\sqrt{n}\sigma}\right\} -\left(1-\mathbb{P}\left\{ \frac{S_{n}}{\sqrt{n}\sigma}\leq\frac{x}{\sqrt{n}\sigma}\right\} \right)\right)\\
 & =2\left(1-\mathbb{P}\left\{ \frac{S_{n}}{\sqrt{n}\sigma}\leq\frac{x}{\sqrt{n}\sigma}\right\} .\right)
\end{align*}
Since we expect $\frac{S_{n}}{\sqrt{n}\sigma}$ to be normally distributed,
letting $F$ denote the distribution function of $\mathcal{N}(0,1)$,
\[
\mathbb{P}\left\{ \left|S_{n}\right|>x\right\} \approx2\left(1-F\left(\frac{x}{\sqrt{n}\sigma}\right)\right).
\]
For example, if $n=10$ and $x=1$, then
\[
\mathbb{P}\left\{ \left|S_{10}\right|>1\right\} \approx2\left(1-F\left(\frac{\sqrt{12}}{\sqrt{10}}\right)\right)\approx0.27.
\]
\end{xca}

\section{Confidence intervals}

The discussion in this section is very hand wavy, so take it with
a grain of salt.

Consider conducting a sequence of trials represented as i.i.d. random
variables $(X_{j})_{j}$. For example, we may repeatedly toss an unfair
coin repeatedly in order to determine its distribution. In practice,
we will not have access to the variance of $X_{1}$.

Letting $S_{n}=X_{1}+\cdots+X_{n}$, and consider the random variable
\[
Z_{n}=\frac{\sqrt{n}}{\sigma}\left(\frac{S_{n}}{n}-\mu\right).
\]
Then,
\[
\mathbb{P}(-z\leq Z_{n}\leq z)=\mathbb{P}\left(\mu-z\frac{\sigma}{\sqrt{n}}\leq\frac{S_{n}}{n}\leq\mu+z\frac{\sigma}{\sqrt{n}}\right).
\]

Now, fix $\alpha\in[0,1]$. Pick $z$ such that
\[
\mathbb{P}(-z\leq Z_{n}\leq z)=1-\alpha.
\]
Since $Z_{n}\xrightarrow{\mathcal{D}}\mathcal{N}(0,1)$,
\begin{align*}
\mathbb{P}(-z\leq Z_{n}\leq z) & =\mathbb{P}(Z_{n}\leq z)-\mathbb{P}(Z_{n}<-z)\\
 & =\mathbb{P}(Z_{n}\leq z)-\left(1-\mathbb{P}(Z_{n}\geq-z)\right)\\
 & \approx\mathbb{P}(Z_{n}\leq z)-\left(1-\mathbb{P}(Z_{n}\leq z)\right)\\
 & =2\mathbb{P}(Z_{n}\leq z)-1
\end{align*}
where we have used the approximation
\[
\mathbb{P}(Z_{n}\geq-z)\approx\mathbb{P}(Z_{n}\leq z)
\]
which is a consequence of $Z_{n}$ being nearly normal. Therefore,
\[
\mathbb{P}(Z_{n}\leq z)\approx1-\frac{\alpha}{2}
\]
and hence
\[
z\approx F^{-1}\left(1-\frac{\alpha}{2}\right).
\]

This tells us that our sample mean $S_{n}/n$ is ``very likely''
to be within 
\[
\pm F^{-1}\left(1-\frac{\alpha}{2}\right)\frac{\sigma}{\sqrt{n}}
\]
of the actual mean. However, this estimate is not so useful since
we may not have access to $\sigma$. So instead, we make another approximation,
substituting the sample variance:
\[
F^{-1}\left(1-\frac{\alpha}{2}\right)\frac{\sigma_{n}}{\sqrt{n}}
\]
where
\[
\sigma_{n}=\frac{1}{n}\sum_{j=1}^{n}\left(X_{j}-\frac{S_{n}}{n}\right)^{2}.
\]
\begin{example}
We toss a coin $100$ times. The result of the $j$-th toss is $X_{j}$.
Suppose we observe $53$ heads so that
\[
\frac{S_{n}(\omega)}{n}=0.53\qquad\text{and}\qquad\sigma_{n}^{2}\approx0.25
\]
where we have written $S_{n}(\omega)$ above to stress that 0.53 is
an observation. Let $\alpha=0.05$, corresponding to a confidence
interval of $1-\alpha=0.95$. Then,
\[
F^{-1}\left(1-\frac{\alpha}{2}\right)=F^{-1}\left(0.975\right)=1.96.
\]
Moreover,
\[
1.96\cdot\frac{\sigma_{n}}{\sqrt{n}}\approx0.1
\]
The confidence interval is
\[
\left[0.53-0.1,0.53+0.1\right]=\left[0.43,0.63\right].
\]
\end{example}

\section{Lyapunov's CLT}

The problem with the classical CLT is that it requires the variables
to be identically distributed. Lyapunov's CLT relaxes this assumption:
\begin{prop}
Let $(X_{j})_{j}$ be a sequence independent, mean zero random variables
such that $X_{j}^{3}$ is integrable. Let $\sigma_{j}^{2}=\operatorname{Var}(X_{j})$,
$\hat{\gamma}_{j}=\mathbb{E}[X_{j}^{3}]$, and $\gamma_{j}=\mathbb{E}[|X_{j}|^{3}]$.
Let $S_{n}=X_{1}+\cdots+X_{n}$ and $s_{n}=\sigma_{1}^{2}+\cdots+\sigma_{n}^{2}$.
Suppose that
\[
\frac{\gamma_{1}+\cdots+\gamma_{n}}{s_{n}^{3}}\rightarrow0\qquad\text{as}\qquad n\rightarrow\infty.
\]
Then,
\[
\frac{S_{n}}{s_{n}}\xrightarrow{\mathcal{D}}\mathcal{N}(0,1)\qquad\text{as}\qquad n\rightarrow\infty.
\]
\end{prop}

\end{document}
