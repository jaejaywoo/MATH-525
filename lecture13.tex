%% LyX 2.2.1 created this file.  For more info, see http://www.lyx.org/.
%% Do not edit unless you really know what you are doing.
\documentclass[english,12pt]{article}
\usepackage[T1]{fontenc}
\usepackage[latin9]{inputenc}
\usepackage{mathtools}
\usepackage{amsmath}
\usepackage{amsthm}
\usepackage{amssymb}

\makeatletter
%%%%%%%%%%%%%%%%%%%%%%%%%%%%%% Textclass specific LaTeX commands.
\theoremstyle{plain}
\newtheorem{thm}{\protect\theoremname}[section]
\theoremstyle{definition}
\newtheorem{example}[thm]{\protect\examplename}
\theoremstyle{definition}
\newtheorem{defn}[thm]{\protect\definitionname}
\theoremstyle{plain}
\newtheorem{prop}[thm]{\protect\propositionname}
\ifx\proof\undefined
\newenvironment{proof}[1][\protect\proofname]{\par
\normalfont\topsep6\p@\@plus6\p@\relax
\trivlist
\itemindent\parindent
\item[\hskip\labelsep\scshape #1]\ignorespaces
}{%
\endtrivlist\@endpefalse
}
\providecommand{\proofname}{Proof}
\fi

%%%%%%%%%%%%%%%%%%%%%%%%%%%%%% User specified LaTeX commands.
\usepackage[margin=1in]{geometry}

\makeatother

\usepackage{babel}
\providecommand{\definitionname}{Definition}
\providecommand{\examplename}{Example}
\providecommand{\propositionname}{Proposition}
\providecommand{\theoremname}{Theorem}

\begin{document}

\title{Math 525: Lecture 13}

\date{February 20, 2018}

\maketitle
For the next few lectures, we will work towards proving the central
limit theorem (CLT). Recall that for a ``sufficiently nice'' sequence
of i.i.d. random variables $(X_{n})_{n}$, the law of large numbers
told us
\[
\frac{S_{n}}{n}\equiv\frac{X_{1}+\cdots+X_{n}}{n}\rightarrow\mathbb{E}X_{1}\text{ a.s.}
\]
The CLT will tell us about the distribution of $S_{n}/n$. Namely,
\[
\sqrt{n}\left(\frac{S_{n}}{n}-\mathbb{E}X_{1}\right)\xrightarrow{\mathcal{D}}Y
\]
where $Y\sim\mathcal{N}(0,\operatorname{Var}(X_{1}))$. To prove the
CLT, we will need to build up a series of results culminating in \emph{Levy's
continuity theorem}.

\section{Convergence in distribution}

We begin with an example:
\begin{example}
Let $X=0$ be a ``deterministic'' random variable with distribution
function $F=[0,\infty)$.
\begin{enumerate}
\item Let $X_{n}=-\frac{1}{n}$. Then, its distribution function is $F_{n}=I_{[1/n,\infty)}$.
Note, in particular, that
\[
F_{n}(x)\rightarrow F(x)
\]
at all points $x$.
\item Now consider $X_{n}=\frac{1}{n}$. Then, its distribution function
is $F_{n}=I_{[\frac{1}{n},\infty)}$. Note, in particular, that
\[
F_{n}(x)\rightarrow F(x)
\]
for all $x$ except $x=0$ since $F_{n}(0)=0$ for all $n$ and $F(0)=1$!
\end{enumerate}
\end{example}
In the second example above, while we expect the distribution function
of $X_{n}$ to converge to that of $X$ (after all, $X_{n}\rightarrow X$
pointwise), it does not! The problem here is the point of discontinuity
of $F$ at $x=0$. This motivates the definition of convergence in
distribution (which we actually saw in a previous lecture).
\begin{defn}
Let $(X_{n})_{n}$ be a sequence of random variables and $X$ be a
random variable. Let $F_{n}$ and $F$ denote the distribution functions
of $X_{n}$ and $X$, respectively. \emph{$(X_{n})_{n}$ converges
to $X$ in distribution} if $F_{n}(x)\rightarrow F(x)$ for all continuity
points of $F$. We write $X_{n}\xrightarrow{\mathcal{D}}X$ or $F_{n}\Rightarrow F$
in this case.
\end{defn}
The next proposition tells us that convergence in probability is stronger
than convergence in distribution (the converse is not true). This
in turn implies that any kind of convergence that we have covered
(a.s. or $L^{p}$) are stronger than convergence in distribution.
\begin{prop}
If $X_{n}\rightarrow X$ in probability, then $X_{n}\xrightarrow{\mathcal{D}}X$.
\end{prop}
\begin{proof}
Let $\epsilon>0$. Trivially,
\begin{align*}
\mathbb{P}\left\{ X_{n}\leq x\right\}  & =\mathbb{P}\left\{ X_{n}\leq x,X\leq x+\epsilon\right\} +\mathbb{P}\left\{ X_{n}\leq x,X>x+\epsilon\right\} \\
 & \leq\mathbb{P}\left\{ X\leq x+\epsilon\right\} +\mathbb{P}\left\{ \left|X_{n}-X\right|>\epsilon\right\} .
\end{align*}
In other words,
\[
F_{n}(x)\leq F(x+\epsilon)+\mathbb{P}\left\{ \left|X_{n}-X\right|>\epsilon\right\} .
\]
Similarly, we can show
\[
F(x-\epsilon)\leq F_{n}(x)+\mathbb{P}\left\{ \left|X_{n}-X\right|>\epsilon\right\} .
\]
Combining these inequalities,
\[
F(x-\epsilon)-\mathbb{P}\left\{ \left|X_{n}-X\right|>\epsilon\right\} \leq F_{n}(x)\leq F(x+\epsilon)+\mathbb{P}\left\{ \left|X_{n}-X\right|>\epsilon\right\} .
\]
Taking $n\rightarrow\infty$,
\[
F(x-\epsilon)\leq\liminf_{n\rightarrow\infty}F_{n}(x)\leq\limsup_{n\rightarrow\infty}F_{n}(x)\leq F(x+\epsilon).
\]
If $x$ is a point of continuity, taking $\epsilon\downarrow0$ yields
$F(x-\epsilon)\rightarrow F(x)$ and $F(x+\epsilon)\rightarrow F(x)$,
and the result is proved.
\end{proof}
Next, we recall a definition from real analysis:
\begin{defn}
Let $A\subset\mathbb{R}$. We say $A$ is \emph{dense} in $\mathbb{R}$
if for each $x\in\mathbb{R}$, we can find a sequence $(x_{n})_{n}$
of elements of $A$ such that $x_{n}\rightarrow x$.
\end{defn}
A classic example is $A=\mathbb{Q}$: any real number can be approximated
as close as desirable by a rational number.
\begin{prop}
\label{prop:convergence}Let $A$ be dense in $\mathbb{R}$. Let $f_{n}\colon\mathbb{R}\rightarrow\mathbb{R}$
and $f\colon\mathbb{R}\rightarrow\mathbb{R}$ be nondecreasing with
$f_{n}(x)\rightarrow f(x)$ for all $x\in A$. Then, $f_{n}(x)\rightarrow f(x)$
at all continuity points of $f$.
\end{prop}
The utility of the above result for us is the case of $f_{n}=F_{n}$
and $f=F$ being distribution functions, which we already know are
nondecreasing.
\begin{proof}
Let $\epsilon>0$. If $x$ is a point of continuity of $f$, we can
find $a,b\in A$ such that
\begin{itemize}
\item $a<x<b$ and
\item $|f(x)-f(y)|<\epsilon$ for all $y\in[a,b]$.
\end{itemize}
Therefore,
\[
\limsup_{n}f_{n}(x)\leq\limsup_{n}f_{n}(b)=f(b)<f(x)+\epsilon
\]
and
\[
\liminf_{n}f_{n}(x)\geq\liminf_{n}f_{n}(a)=f(a)>f(x)-\epsilon.
\]
In summary,
\[
f(x)-\epsilon<\liminf_{n}f_{n}(x)\leq\limsup_{n}f_{n}(x)<f(x)+\epsilon.
\]
Taking $\epsilon\downarrow0$, we get equality.
\end{proof}
We next visit an alternate characterization of convergence in distribution.
Recall that $f\colon\mathbb{R}\rightarrow\mathbb{R}$ is said to be
bounded if we can find $M>0$ such that $|f(x)|\leq M$ for all $x\in\mathbb{R}$.
\begin{prop}
$X_{n}\xrightarrow{\mathcal{D}}X$ if and only if for all bounded
functions $f\colon\mathbb{R}\rightarrow\mathbb{R}$,
\begin{equation}
\mathbb{E}\left[f(X_{n})\right]\rightarrow\mathbb{E}\left[f(X)\right].\label{eq:limit}
\end{equation}
\end{prop}
\begin{proof}[Proof ($\Rightarrow$)]
 Suppose $X_{n}\xrightarrow{\mathcal{D}}X$. We start by establishing
(\ref{eq:limit}) for all continuous functions $f$ with compact support
(i.e., $f(x)=0$ if $|x|\geq r$ for some $r>0$). In this case, $f$
is uniformly continuous.

Now, let $\epsilon>0$. Consider a partition $\{(x_{i-1},x_{i}]\}_{i=1}^{n}$
of $(-r,r]$ satisfying
\[
\left|f(x)-f(x_{i})\right|<\epsilon/2\qquad\text{for all }x\in(x_{i-1},x_{i}]
\]
and chosen such that each $x_{i}$ is a point of continuity of $F$,
the distribution function of $X$. Let $\psi$ be the function defined
by
\[
\psi(x)=\sum_{i=1}^{n}f(x_{i})I_{(x_{i-1},x_{i}]}(x).
\]
Note, in particular, that
\[
\left|f(x)-\psi(x)\right|\leq\sum_{i=1}^{n}\left|f(x)-f(x_{i})\right|I_{(x_{i-1},x_{i}]}(x)<\epsilon/2.
\]
Recalling that any bounded random variable is trivially integrable,
we have
\[
\left|\mathbb{E}f(X)-\mathbb{E}f(X_{n})\right|\leq\left|\mathbb{E}f(X)-\mathbb{E}\psi(X)\right|+\left|\mathbb{E}\psi(X)-\mathbb{E}\psi(X_{n})\right|+\left|\mathbb{E}\psi(X_{n})-\mathbb{E}f(X_{n})\right|.
\]
The first and last expectations are bounded by $\epsilon/2$. As for
the middle expectation,
\begin{align*}
\left|\mathbb{E}\psi(X)-\mathbb{E}\psi(X_{n})\right| & =\left|\mathbb{E}\left[\psi(X)-\psi(X_{n})\right]\right|\\
 & =\left|\sum_{i=1}^{n}f(x_{i})\mathbb{E}\left[I_{(x_{i-1},x_{i}]}(X)-I_{(x_{i-1},x_{i}]}(X_{n})\right]\right|\\
 & =\left|\sum_{i=1}^{n}f(x_{i})\left[\left(F(x_{i})-F(x_{i-1})\right)-\left(F_{n}(x_{i})-F_{n}(x_{i-1})\right)\right]\right|
\end{align*}
Now, since each $x_{i}$ is a point of continuity of $F$, we have
$F_{n}(x_{i})\rightarrow F(x_{i})$ for all $i$. Therefore,
\[
\limsup_{n}\left|\mathbb{E}f(X)-\mathbb{E}f(X_{n})\right|<\epsilon.
\]
Now take $\epsilon\downarrow0$ to get
\[
\lim_{n}\left|\mathbb{E}f(X)-\mathbb{E}f(X_{n})\right|=0.
\]

Now, let's see how to modify the proof if $f$ is just bounded and
continuous but not necessarily of compact support. Let $\epsilon>0$
and choose $K$ such that $\pm K$ are continuity points of $F$,
$F(-K)<\epsilon/4$ and $F(K)>1-\epsilon/4$ (remember, $F$ is a
distribution function, so that $F(-x)\rightarrow0$ and $F(x)\rightarrow1$
as $x\rightarrow\infty$). Let $g$ be a continuous function such
that $g(x)=1$ if $|x|\leq K$ and $g(x)=0$ if $|x|\geq K+1$. Then,
$fg$ is continuous, equals $f$ on $[-K,K]$, and has compact support.
Therefore,
\begin{multline}
\left|\mathbb{E}f(X)-\mathbb{E}f(X_{n})\right|\leq\Vert f\Vert_{\infty}\left(\mathbb{P}\left\{ X\leq-K\right\} +\mathbb{P}\left\{ X_{n}\leq-K\right\} \right)\\
+\Vert f\Vert_{\infty}\left(\mathbb{P}\left\{ X>K\right\} +\mathbb{P}\left\{ X_{n}>K\right\} \right)+\left|\mathbb{E}\left[f(X)g(X)\right]-\mathbb{E}\left[f(X_{n})g(X_{n})\right]\right|.\label{eq:bounds}
\end{multline}
Note that
\[
\mathbb{P}\left\{ X\leq-K\right\} +\mathbb{P}\left\{ X_{n}\leq-K\right\} =F(-K)+F_{n}(-K)\rightarrow2F(-K)<\epsilon/2
\]
Similarly, 
\[
\mathbb{P}\left\{ X>K\right\} +\mathbb{P}\left\{ X_{n}>K\right\} =1-F(K)+1-F_{n}(K)\rightarrow2\left(1-F(K)\right)<\epsilon/2.
\]
Therefore, the sum of the first two terms on the right hand side of
(\ref{eq:bounds}) is bounded by $\epsilon\Vert f\Vert_{\infty}$.
The last term is handled as in the previous paragraph, since $fg$
has compact support.
\end{proof}
%
\begin{proof}[Proof ($\Leftarrow$)]
 Suppose (\ref{eq:limit}) holds for all bounded and continuous functions
$f$. Let $a$ be a point of continuity of $F$. Let $f_{m}$ and
$g_{m}$ be continuous and bounded functions such that
\[
f_{m}(x)=\begin{cases}
1 & \text{if }x\leq a-1/m\\
0 & \text{if }x\geq a
\end{cases}
\]
and
\[
g_{m}(x)=\begin{cases}
1 & \text{if }x\leq a\\
0 & \text{if }x\geq a+1/m.
\end{cases}
\]
Moreover, suppose $f_{m}$ and $g_{m}$ are linear on $[a-1/m,a]$
and $[a,a+1/m]$, respectively.

Note that $f_{m}\leq I_{(-\infty,a]}\leq g_{m}$. Therefore,
\[
\mathbb{E}\left[f_{m}(X_{n})\right]\leq F_{n}(a)\leq\mathbb{E}\left[g_{m}(X_{n})\right]
\]
since
\[
\mathbb{E}\left[I_{(-\infty,a]}(X_{n})\right]=\mathbb{P}\left\{ X_{n}\leq a\right\} =F_{n}(a).
\]
Now, by (\ref{eq:limit}), we can take $n\rightarrow\infty$ to get
$\mathbb{E}[f_{m}(X_{n})]\rightarrow\mathbb{E}[f_{m}(X)]$ and $\mathbb{E}[g_{m}(X_{n})]\rightarrow\mathbb{E}[g_{m}(X)]$.
Therefore,
\[
\mathbb{E}\left[f_{m}(X)\right]\leq\liminf_{n}F_{n}(a)\leq\limsup_{n}F_{n}(a)\leq\mathbb{E}\left[g_{m}(X)\right].
\]
Moreover, as $m\rightarrow\infty$, $f_{m}\rightarrow I_{(-\infty,a)}$
and $g_{m}\rightarrow I_{(-\infty,a]}$. By the DCT,
\[
\mathbb{E}\left[f_{m}(X)\right]\rightarrow\mathbb{E}\left[I_{(-\infty,a)}(X)\right]=\mathbb{P}\left\{ X<a\right\} =F(a-)
\]
and
\[
\mathbb{E}\left[g_{m}(X)\right]\rightarrow\mathbb{E}\left[I_{(-\infty,a]}(X)\right]=\mathbb{P}\left\{ X\leq a\right\} =F(a).
\]
But $a$ is a continuity point of $F$, and hence $F(a-)=F(a)$, from
which we get the desired result.
\end{proof}

\end{document}
