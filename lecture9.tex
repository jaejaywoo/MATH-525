%% LyX 2.2.1 created this file.  For more info, see http://www.lyx.org/.
%% Do not edit unless you really know what you are doing.
\documentclass[english,12pt]{article}
\usepackage[T1]{fontenc}
\usepackage[latin9]{inputenc}
\usepackage{float}
\usepackage{mathtools}
\usepackage{amsmath}
\usepackage{amsthm}
\usepackage{amssymb}
\usepackage{graphicx}

\makeatletter
%%%%%%%%%%%%%%%%%%%%%%%%%%%%%% Textclass specific LaTeX commands.
\theoremstyle{plain}
\newtheorem{thm}{\protect\theoremname}[section]
\theoremstyle{definition}
\newtheorem{defn}[thm]{\protect\definitionname}
\theoremstyle{plain}
\newtheorem{prop}[thm]{\protect\propositionname}
\ifx\proof\undefined
\newenvironment{proof}[1][\protect\proofname]{\par
\normalfont\topsep6\p@\@plus6\p@\relax
\trivlist
\itemindent\parindent
\item[\hskip\labelsep\scshape #1]\ignorespaces
}{%
\endtrivlist\@endpefalse
}
\providecommand{\proofname}{Proof}
\fi
\theoremstyle{definition}
\newtheorem{example}[thm]{\protect\examplename}

%%%%%%%%%%%%%%%%%%%%%%%%%%%%%% User specified LaTeX commands.
\usepackage[margin=1in]{geometry}

\makeatother

\usepackage{babel}
\providecommand{\definitionname}{Definition}
\providecommand{\examplename}{Example}
\providecommand{\propositionname}{Proposition}
\providecommand{\theoremname}{Theorem}

\begin{document}

\title{Math 525: Lecture 9}

\date{February 1, 2018}
\maketitle

\section{Convergence of random variables}

In a previous lecture, we saw that (roughly speaking) $B(n,\frac{\lambda}{n})\rightarrow\operatorname{Poisson}(\lambda)$
as $n\rightarrow\infty$. But what does ``$\rightarrow$'' mean
here? More generally, consider a sequence of random variables $X_{1},X_{2},\ldots$
What does it mean for this sequence to converge? Are there multiple
notions of convergence available? Which notions of convergence are
useful? This will be the topic of today's lecture.
\begin{defn}
The sequence $(X_{n})_{n}$ of random variables \emph{converges pointwise}
to a random variable $X$ if $X_{n}(\omega)\rightarrow X(\omega)$
for each $\omega$ in the sample space.
\end{defn}
This is the usual definition of pointwise convergence for functions,
but it's not the most useful in probability. Why? Consider two random
variables $X$ and $Y$ which are equal to a.s., but we can find $\omega$
such that $X(\omega)\neq Y(\omega)$. Then, the alternating sequence
$X,Y,X,Y,\ldots$ does not have a pointwise limit, even though the
two random variables are ``essentially'' the same!

The above discussion implies that we need a weaker notion of convergence
when it comes to probability:
\begin{defn}
Let $(X_{n})_{n}$ be a sequence of random variables and $X$ be a
random variable.
\begin{enumerate}
\item $(X_{n})_{n}$\emph{ }converges\emph{ in probability} to $X$ if for
all $\epsilon>0$,
\[
\mathbb{P}\left\{ \left|X_{n}-X\right|>\epsilon\right\} \rightarrow0\text{ as }n\rightarrow\infty.
\]
\item $(X_{n})_{n}$ converges to $X$ \emph{with probability one} or \emph{almost
everywhere} (a.e.) if $X_{n}(\omega)\rightarrow X(\omega)$ for all
$\omega\notin\Lambda$ where $\mathbb{P}(\Lambda)=0$.
\item $(X_{n})_{n}$ converges to $X$ in $\mathbb{L}^{p}$ if $X^{p}$
is integrable and
\[
\mathbb{E}\left[\left|X_{n}-X\right|^{p}\right]\rightarrow0\text{ as }n\rightarrow\infty.
\]
We write $X_{n}\xrightarrow{\mathbb{L}^{p}}X$ in this case. When
$p=1$, we call this ``convergence in mean''.
\item Let $F_{n}$ and $F$ denote the distribution functions of $X_{n}$
and $X$, respectively. $(X_{n})_{n}$ converges to $X$ in distribution
if $F_{n}(x)\rightarrow F(x)$ for all continuity points of $F$.
We write $X_{n}\xrightarrow{\mathcal{D}}X$ in this case.
\end{enumerate}
\end{defn}
Some notions of convergence are stronger than others:
\begin{prop}
If $X_{n}\xrightarrow{\mathbb{L}^{p}}X$, then $X_{n}\rightarrow X$
in probability.
\end{prop}
\begin{proof}
This is a consequence of Chebyshev's inequality:
\[
\mathbb{P}\left\{ \left|X_{n}-X\right|>\epsilon\right\} \leq\frac{1}{\epsilon^{p}}\mathbb{E}\left[\left|X_{n}-X\right|^{p}\right]\rightarrow0.\qedhere
\]
\end{proof}
\begin{prop}
If $X_{n}\rightarrow X$ a.e., then $X_{n}\rightarrow X$ in probability.
\end{prop}
\begin{proof}
Suppose $X_{n}\rightarrow X$ pointwise for all $\omega\notin\Lambda$
where $\mathbb{P}(\Lambda)=1$. Let
\[
Z_{n}=\sup_{k\geq n}\left|X_{k}-X\right|
\]
and note that $\lim_{n}Z_{n}=\limsup_{n}|X_{n}-X|$. Therefore, 
\[
X_{n}(\omega)\rightarrow X(\omega)\iff Z_{n}(\omega)\rightarrow0.
\]
Let $\epsilon>0$ and 
\[
\Gamma_{n}^{\epsilon}=\left\{ Z_{n}\geq\epsilon\right\} .
\]
If $\omega\in\cap_{n}\Gamma_{n}^{\epsilon}$, then $Z(\omega)\nrightarrow0$,
and hence $\cap_{n}\Gamma_{n}^{\epsilon}\subset\Lambda$. Moreover,
note that these sets are decreasing in containment: 
\[
\Gamma_{1}^{\epsilon}\supset\Gamma_{2}^{\epsilon}\supset\cdots
\]
Therefore, $\mathbb{P}(\Gamma_{n}^{\epsilon})\rightarrow\mathbb{P}(\cap_{n}\Gamma_{n}^{\epsilon})\leq\mathbb{P}(\Lambda)=0$.
Since $|X_{n}-X|\leq Z_{n}$,
\[
\mathbb{P}\left\{ \left|X_{n}-X\right|\geq\epsilon\right\} \leq\mathbb{P}(\Gamma_{n}^{\epsilon})\rightarrow0.\qedhere
\]
\end{proof}
The converse of the above is not true in general:
\begin{example}
Let $Y\sim U[0,1]$ and define
\begin{align*}
X_{1} & =1\\
X_{2} & =I_{[0,1/2]}(Y)\\
X_{3} & =I_{(1/2,1]}(Y)\\
X_{4} & =I_{[0,1/4]}(Y)\\
X_{5} & =I_{(1/4,1/2]}(Y)\\
\vdots
\end{align*}
Note that $X_{n}\rightarrow0$ in probability. However, there is no
$\omega$ for which $X_{n}(\omega)\rightarrow0$!
\end{example}
It is not trivial that a limit of a sequence of random variables is
a random variable itself, so we prove that next. As a technical point,
since limits may introduce values of $-\infty$ and $+\infty$, we
need to work with random variables which can take on infinite values:
\begin{defn}
Let $\overline{\mathbb{R}}=\mathbb{R}\cup\{-\infty,+\infty\}$ denote
the \emph{extended real line}. An \emph{extended real valued (ERV)
random variable} is a function $X\colon\Omega\rightarrow\overline{\mathbb{R}}$
such that
\[
\left\{ X\leq x\right\} \in\mathcal{F}\qquad\text{for all }x\in\overline{\mathbb{R}}.
\]
\end{defn}
\begin{prop}
Let $X_{1},X_{2},\ldots$ be ERV random variables. Then, $M$, $m$,
and $X_{\infty}$ are also ERV random variables where
\begin{enumerate}
\item $M(\omega)=\sup X_{n}(\omega)$.
\item $m(\omega)=\limsup_{n\rightarrow\infty}X_{n}(\omega)$.
\item $X_{\infty}(\omega)=\begin{cases}
\lim_{n}X_{n}(\omega) & \text{if the limit exists}\\
0 & \text{otherwise}.
\end{cases}$
\end{enumerate}
\end{prop}
Note that by taking the negation of the first two, we find that $\inf X_{n}$
and $\liminf_{n\rightarrow\infty}X_{n}$ are also ERV random variables.
\begin{proof}
~
\begin{enumerate}
\item We need to show that $\{\omega\colon M(\omega)\leq x\}=\cap_{n}\{\omega\colon X_{n}(\omega)\leq x\}$
is in $\mathcal{F}$ for any $x\in\overline{\mathbb{R}}$. Since it
is a countable intersection of sets in $\mathcal{F}$, the desired
result follows.
\item Note that $m=\inf_{n}Y_{n}$ where $Y_{n}=\sup_{k\geq n}X_{k}$. We
know by the previous point that $Y_{n}$ is an ERV random variable
for each $n$. Therefore, $\sup_{n}-Y_{n}=-\inf Y_{n}$ is an ERV
random variable, and so too is $m$.
\item Let 
\[
\Lambda_{\infty}=\left\{ \omega\colon\limsup_{n}X_{n}(\omega)=\liminf_{n}X_{n}(\omega)\right\} .
\]
This set is in $\mathcal{F}$, and hence we can define the random
variable
\[
Y_{n}=I_{\Lambda_{\infty}}X_{n}.
\]
The desired result follows because
\[
\limsup_{n}Y_{n}=\lim_{n}X_{n}
\]
is an extended real-valued random variable.\qedhere
\end{enumerate}
\end{proof}

\section{Borel-Cantelli lemma}
\begin{defn}
Let $(\Lambda_{n})_{n}$ be a sequence of subsets of $\Omega$. Define
\begin{align*}
\limsup_{n}\Lambda_{n} & =\left\{ \omega\mid\forall N\colon\exists n\geq N\colon\omega\in\Lambda_{n}\right\} \\
\liminf_{n}\Lambda_{n} & =\left\{ \omega\mid\exists N\colon\forall n\geq N\colon\omega\in\Lambda_{n}\right\} .
\end{align*}
\end{defn}
Intuitively, we can think of $\limsup_{n}\Lambda_{n}$ as the set
of all $\omega$ such that $\omega\in\Lambda_{n}$ for ``infinitely
many $n$''. Similarly, $\liminf_{n}\Lambda_{n}$ is the set of all
$\omega$ such that $\omega\in\Lambda_{n}$ for ``all but finitely
many $n$''. Expressed in an equivalent way,
\begin{align*}
\limsup_{n}\Lambda_{n} & =\bigcap_{N}\bigcup_{n\geq N}\Lambda_{m}\\
\liminf_{n}\Lambda_{n} & =\bigcup_{N}\bigcap_{n\geq N}\Lambda_{m}.
\end{align*}
Applying De Morgan's law, we see
\[
\left(\limsup\Lambda_{n}\right)^{c}=\left(\bigcap_{N}\bigcup_{n\geq N}\Lambda_{m}\right)^{c}=\bigcup_{N}\bigcap_{n\geq N}\Lambda_{m}^{c}=\liminf\Lambda_{n}^{c}
\]
and hence $\limsup\Lambda_{n}^{c}=(\liminf\Lambda_{n})^{c}$ also.
\begin{prop}[Borel-Cantelli lemma]
Let $(\Lambda_{n})_{n}$ be a sequence in $\mathcal{F}$. Suppose
$\sum_{n}\mathbb{P}(\Lambda_{n})<\infty$. Then,
\[
\mathbb{P}\left\{ \limsup\Lambda_{n}\right\} =0.
\]
\end{prop}
\begin{proof}
Recall that 
\[
\limsup_{n}\Lambda_{n}=\bigcap_{N}A_{N}\qquad\text{where}\qquad A_{N}=\bigcup_{n\geq N}\Lambda_{n}.
\]
In particular, $(A_{N})_{N}$ is decreasing in containment: $A_{1}\supset A_{2}\supset\cdots$
Therefore,
\[
\lim_{N\rightarrow\infty}\mathbb{P}(A_{N})=\mathbb{P}\left(\bigcap_{N}A_{N}\right)=\mathbb{P}\left(\limsup_{n}\Lambda_{n}\right)\qquad\text{as}\qquad N\rightarrow\infty.
\]
But note that
\[
\lim_{N\rightarrow\infty}\mathbb{P}(A_{N})=\lim_{N\rightarrow\infty}\mathbb{P}\left(\bigcup_{n\geq N}\Lambda_{n}\right)\leq\lim_{N\rightarrow\infty}\sum_{n\geq N}\Lambda_{n}\rightarrow0.\qedhere
\]
\end{proof}
Next, we look at an important application of the Borel-Cantelli lemma:
\begin{prop}
\label{prop:extract}Let $(X_{n})_{n}$ be a sequence of ERV random
variables and suppose $X_{n}\rightarrow X$ in probability. Then,
there exists a subsequence $(n_{k})_{k}$ such that $X_{n_{k}}\rightarrow X$
a.e.
\end{prop}
\begin{proof}
Taking $\epsilon=1/m$ in the definition of convergence in probability,
we find that for each $m$,
\[
\mathbb{P}\left\{ \left|X_{n}-X\right|>1/m\right\} \rightarrow0\qquad\text{as }n\rightarrow\infty.
\]
Choose an increasing sequence $(n_{m})_{m}$ such that
\[
\mathbb{P}(\Lambda_{m})<1/2^{m}.
\]
where
\[
\Lambda_{m}=\left\{ \left|X_{n_{m}}-X\right|>\frac{1}{m}\right\} .
\]
Then, $\sum_{m}\mathbb{P}(\Lambda_{m})<\infty$, and therefore $\mathbb{P}(\limsup_{m}\Lambda_{m})=0$
by the Borel-Cantelli lemma. Therefore,
\[
\mathbb{P}\left(\liminf_{m}\Lambda_{m}^{c}\right)=1-\mathbb{P}\left(\limsup_{m}\Lambda_{m}\right)=1.
\]

Recall now that
\[
\liminf_{m}\Lambda_{m}^{c}=\left\{ \omega\mid\exists N\colon\forall m\geq N\colon\omega\notin\Lambda_{n}\right\} .
\]
Therefore, for each $\omega\in\liminf_{m}\Lambda_{m}$, we can find
an $N(\omega)$ (depending on $\omega$) such that for all $m\geq N(\omega)$,
we have $\omega\notin\Lambda_{m}$. It follows that for $\omega\in\liminf_{m}\Lambda_{m}$,
\[
\left|X_{n_{m}}(\omega)-X(\omega)\right|\leq\frac{1}{m}\qquad\text{for all }m\geq N(\omega)
\]
and hence
\[
\left|X_{n_{m}}(\omega)-X(\omega)\right|\rightarrow0.
\]
That is, $X_{n_{m}}\rightarrow X$ pointwise on the set $\liminf_{m}\Lambda_{m}$,
which is exactly the definition of a.e. convergence.
\end{proof}

\section{Modes of convergence diagram}

The following diagram summarizes convergence relationships in a probability
space. AE refers to a.e. convergence, $L^{p}$ refers to what we've
been calling $\mathbb{L}^{p}$ convergence, and $M$ refers to convergence
in probability. AU is a.u. convergence which we will most likely not
encounter, though it's definition is below for those who are interested.

\begin{figure}[H]
\begin{centering}
\includegraphics[width=3in]{modes}
\par\end{centering}
\caption{Modes of convergence (diagram by John Cook). A solid line from one
mode of convergence to another indicates implication (e.g., a.e. convergence
implies convergence in probability). A dashed line means that we can
extract a subsequence, as in Proposition \ref{prop:extract}.}
\end{figure}


\subsection{Almost uniform convergence (optional)}

Let $(f_{n})_{n}$ be a sequence of real-valued functions and $f$
be a real-valued function such that $f_{n}$ and $f$ have the same
domain. We say $f_{n}\rightarrow f$ \emph{uniformly} if for each
$\epsilon>0$, we can find $N$ such that for all $n\geq N$,
\[
\left\Vert f_{n}-f\right\Vert _{\infty}\equiv\sup_{x}\left|f_{n}(x)-f(x)\right|<\epsilon.
\]

\begin{defn}
Let $(X_{n})_{n}$ be a sequence of random variables and $X$ be a
random variable. We say $X_{n}\rightarrow X$ almost uniformly (a.u.)
if for every $\epsilon>0$, we can find a set $A\in\mathcal{F}$ with
$\mathbb{P}(A)<\epsilon$ such that $X_{n}\mid_{A}\rightarrow X\mid_{A}$
converges uniformly. Here, the symbol $\cdot\mid_{A}$ is the restriction
of a function to the set $A$.
\end{defn}

\end{document}
