%% LyX 2.2.3 created this file.  For more info, see http://www.lyx.org/.
%% Do not edit unless you really know what you are doing.
\documentclass[english,12pt]{article}
\usepackage[T1]{fontenc}
\usepackage[latin9]{inputenc}
\usepackage{color}
\usepackage{babel}
\usepackage{mathtools}
\usepackage{amsmath}
\usepackage{amssymb}
\usepackage[unicode=true,pdfusetitle,
 bookmarks=true,bookmarksnumbered=false,bookmarksopen=false,
 breaklinks=false,pdfborder={0 0 1},backref=false,colorlinks=true]
 {hyperref}

\makeatletter
%%%%%%%%%%%%%%%%%%%%%%%%%%%%%% User specified LaTeX commands.
\usepackage[margin=1in]{geometry}

\makeatother

\begin{document}

\title{Math 525: Assignment 3}

\date{\date{}}
\maketitle
\begin{enumerate}
\item (Borel measurable) Let $f\colon\mathbb{R}\rightarrow\mathbb{R}$ be
continuous.
\begin{enumerate}
\item Let 
\[
\mathcal{G}=\left\{ (a,b)\colon-\infty<a<b<\infty\right\} 
\]
be the set of all open intervals. Show $\sigma(\mathcal{G})=\mathcal{B}(\mathbb{R})$.\footnote{\textbf{Hint}: use the fact that $(x,\infty)=(x,x+2)\cup(x+1,x+3)\cup\cdots$
and $(-\infty,x]=\mathbb{R}\setminus(x,\infty)$}
\item Define
\[
\mathcal{M}=\left\{ B\subset\mathbb{R}\colon f^{-1}(B)\in\mathcal{B}(\mathbb{R})\right\} .
\]
Show that $\mathcal{M}$ is a $\sigma$-algebra.\footnote{\textbf{Hint}: use the properties of $f^{-1}$ from the previous assignment}
This establishes $\sigma(\mathcal{M})=\mathcal{M}$.
\item (Optional) Show that for any $G\in\mathcal{G}$, $f^{-1}(G)$ is a
countable union of open intervals. This establishes $\mathcal{G}\subset\mathcal{M}$.
\item Use (a), (b), and (c) to conclude that $\mathcal{B}(\mathbb{R})\subset\mathcal{M}$
and hence $f$ is Borel measurable.
\end{enumerate}
\item (Distribution function) Let $X$ be a discrete random variable with
distribution function $F$. Show that $\sum_{n}F(x_{n})-F(x_{n}-)=1$.
\item (Uniform random variable) Define the function $f\colon\mathbb{R}\rightarrow\mathbb{R}$
by 
\[
f(x)=\begin{cases}
\lambda e^{-\lambda x} & \text{if }x\geq0\\
0 & \text{otherwise}
\end{cases}
\]
and the function $F\colon\mathbb{R}\rightarrow[0,1]$ by $F(x)=\int_{-\infty}^{x}f(y)dy$.
Let $Y\sim U[0,1]$. Note that $F^{-1}(\{0\})$ is a set contaning
more than one element. Moreover, $F^{-1}(\{1\})$ is empty. As such,
$F$ is not technically a bijection, and hence the expression $F^{-1}(Y)$
is not well-defined. However, note that $\{Y=0\}$ and $\{Y=1\}$
occur with zero probability. Therefore, assuming $Y$ does not take
the values $0$ or $1$, we can unambiguously define $X=F^{-1}(Y)$.
\begin{enumerate}
\item Simplify the expression $X=F^{-1}(Y)$ as much as possible.
\item What is the distribution function of $X$?
\end{enumerate}
\item (Expectation) Let $X_{1},X_{2},\ldots$ be nonnegative integer-valued
random variables. Suppose they are independent and have the same distribution
function $F$ (in this case, we say $X_{1},X_{2},\ldots$ are \emph{independent
and identically distributed}, or i.i.d. for short). Show that\textbf{}\footnote{\textbf{Hint}: use the fact that $\mathbb{E}Y=\sum_{n\geq1}\mathbb{P}(\{Y\geq n\})$
for a nonnegative integer-valued random variable $Y$}
\[
\mathbb{E}\min\{X_{1},\ldots,X_{m}\}=\sum_{n=1}^{\infty}\mathbb{P}\{X_{1}\geq n\}^{m}.
\]
\item (Integrability) Let $X$ be a discrete random variable. Suppose $X^{2}$
is integrable. Show that $X$ is integrable.\textbf{}\footnote{\textbf{Hint}: use the inequality $|a|\leq\max\{1,|a|^{2}\}$}
\end{enumerate}

\end{document}
