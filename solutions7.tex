%% LyX 2.2.1 created this file.  For more info, see http://www.lyx.org/.
%% Do not edit unless you really know what you are doing.
\documentclass[english,12pt]{article}
\usepackage[T1]{fontenc}
\usepackage[latin9]{inputenc}
\usepackage{color}
\usepackage{babel}
\usepackage{mathtools}
\usepackage{amsmath}
\usepackage{amssymb}
\usepackage[unicode=true,pdfusetitle,
 bookmarks=true,bookmarksnumbered=false,bookmarksopen=false,
 breaklinks=false,pdfborder={0 0 1},backref=false,colorlinks=true]
 {hyperref}
\usepackage{breakurl}

\makeatletter
%%%%%%%%%%%%%%%%%%%%%%%%%%%%%% User specified LaTeX commands.
\usepackage[margin=1in]{geometry}

\makeatother

\begin{document}

\title{Math 525: Assignment 7 Solutions}

\date{\date{}}
\maketitle
\begin{enumerate}
\item We know $X_{n},Y_{n}$ converge in distribution to $X,Y\sim\operatorname{Poisson}(\lambda)$.
Therefore,
\[
\phi_{Z_{n}}(t)=\mathbb{E}\left[e^{it\left(X_{n}-Y_{n}\right)}\right]=\mathbb{E}\left[e^{itX_{n}}\right]\mathbb{E}\left[e^{-itY_{n}}\right]=\phi_{X_{n}}(t)\phi_{Y_{n}}(-t)\rightarrow\phi_{X}(t)\phi_{Y}(-t)
\]
by one direction of L�vy's continuity theorem. Therefore,
\[
\phi_{X}(t)\phi_{Y}(-t)=\exp(\lambda(e^{it}-1))\exp(\lambda(e^{-it}-1))=\exp\left(\lambda\left(e^{it}-e^{-it}\right)\right)=\exp\left(2\lambda\left(\cosh t-1\right)\right).
\]
By the other direction of L�vy's continuity theorem, $Z_{n}$ converges
in distribution to some random variable $Z$ with the above characteristic
function $\phi_{X}\phi_{Y}$, as desired.
\item See Exercise 2.2 of Lecture 15.
\item To show that $((X_{n},X_{n+1}))_{n\geq0}$ is a Markov chain, note
that
\begin{multline*}
\mathbb{P}\left((X_{n},X_{n+1})=(i_{n},i_{n+1})\mid(X_{0},X_{1})=(i_{0},i_{1}),\ldots,(X_{n-1},X_{n})=(i_{n-1},i_{n})\right)\\
=\mathbb{P}\left(X_{n+1}=i_{n+1}\mid X_{0}=i_{0},\ldots,X_{n}=i_{n}\right)=\mathbb{P}\left(X_{n+1}=i_{n+1}\mid X_{n}=i_{n}\right)\\
=\mathbb{P}\left(X_{n+1}=i_{n+1}\mid X_{n-1}=i_{n-1},X_{n}=i_{n}\right)\\
=\mathbb{P}\left((X_{n},X_{n+1})=(i_{n},i_{n+1})\mid(X_{n-1},X_{n})=(i_{n-1},i_{n})\right).
\end{multline*}
This follows from
\item ~
\begin{enumerate}
\item If $P$ is a transition matrix, then
\[
\sum_{j}(P^{2})_{ij}=\sum_{j}\sum_{k}P_{ik}P_{kj}=\sum_{k}\sum_{j}P_{ik}P_{kj}=\sum_{k}P_{ik}\sum_{j}P_{kj}=\sum_{k}P_{ik}=1.
\]
\item If $P$ is a bistochastic matrix, then $P$ and $P^{\intercal}$ are
transition matrices. Then, by part (a), $P^{2}$ and $(P^{\intercal})^{2}=(P^{2})^{\intercal}$
are transition matrices. Therefore, $P^{2}$ is bistochastic.
\end{enumerate}
\item ~
\begin{enumerate}
\item The multiplicity of the eigenvalue is 2.
\item The multiplicity of the eigenvalue is 1.
\item An ``end'' state of the game is called an \emph{absorbing state.
}Since there is a walk from an any state $u$ to an absorbing state
$v$, the multiplicity of the eigenvalue 1 tells us how many absorbing
states there are.
\end{enumerate}
\end{enumerate}

\end{document}
