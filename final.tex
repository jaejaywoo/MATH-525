%% LyX 2.2.1 created this file.  For more info, see http://www.lyx.org/.
%% Do not edit unless you really know what you are doing.
\documentclass[english,12pt,addpoints]{exam}
\usepackage[T1]{fontenc}
\usepackage[latin9]{inputenc}
\setcounter{secnumdepth}{2}
\setcounter{tocdepth}{2}
\usepackage{fancybox}
\usepackage{calc}
\usepackage{mathtools}
\usepackage{amsmath}
\usepackage{amssymb}
\PassOptionsToPackage{normalem}{ulem}
\usepackage{ulem}

\makeatletter
%%%%%%%%%%%%%%%%%%%%%%%%%%%%%% Textclass specific LaTeX commands.
%% Map the lyx-friendly SmartQuestion style to exam environments
%% See http://tex.stackexchange.com/a/319996/17868
\newcounter{mydepth}%
\newcommand{\myenvironmentname}{%
\ifnum\value{mydepth}=0notinquestion\fi%
\ifnum\value{mydepth}=1question\fi%
\ifnum\value{mydepth}=2part\fi%
\ifnum\value{mydepth}=3subpart\fi%
\ifnum\value{mydepth}=4subsubpart\fi%
\ifnum\value{mydepth}>4undefinedquestion\fi%
}%
\newcommand{\myquestion}{\csname \myenvironmentname\endcsname}%
\newenvironment{myquestions}%
  {\stepcounter{mydepth}\begin{\myenvironmentname s}}%
  {\end{\myenvironmentname s}\addtocounter{mydepth}{-1}}%

% code to rewrite \mychoice or \mychoice[] to \choice or \CorrectChoice respectively
% see http://tex.stackexchange.com/a/319470/17868
\def\mychoice{\@ifnextchar[{\@with}{\@without}}
\def\@with[#1]{\CorrectChoice}
\def\@without{\choice}


\makeatother

\usepackage{babel}
\begin{document}

\footer{}{}{April 2018}{}

\title{Final Exam}

\date{\date{}}

\author{Math 525: Probability}

\maketitle
\thispagestyle{headandfoot} % hack to work around \maketitle clobbering the first page's headers/footers
\begin{center}
\noindent\begin{minipage}[t]{1\textwidth}%
Last name, first name:\enskip{}\hrulefill{}~%
\end{minipage}\vspace{0.2in}
\noindent\begin{minipage}[t]{1\textwidth}%
Section number:\enskip{}\hrulefill{}~%
\end{minipage}\vspace{0.2in}
\noindent\begin{minipage}[t]{1\textwidth}%
User ID:\enskip{}\hrulefill{}~%
\end{minipage}\vfill{}
\doublebox{\begin{minipage}[t]{5.5in}%
\begin{center}
Answer the questions in the spaces provided on the question sheets.
If you run out of room for an answer, continue on the back of the
page.
\par\end{center}%
\end{minipage}}
\par\end{center}

\begin{center}
\doublebox{\begin{minipage}[t]{5.5in}%
\begin{center}
If you are unsure about a question, leave it blank (or cross out any
work you have done on it), to be awarded 25\% of its points.
\par\end{center}%
\end{minipage}}
\par\end{center}

\noindent \begin{flushleft}
\qformat{\textbf{Question \thequestion}\quad (\totalpoints\ \points)\hfill}
\pointsinrightmargin
\printanswers
\par\end{flushleft}

\noindent \newpage{}\vspace{0.1in}

\begin{myquestions}
\myquestion[5] Let $\mathbb{P}$ be a probability measure and $A$ and $B$ be events.
Show that 
\[
\mathbb{P}(A\cup B)=\mathbb{P}(A)+\mathbb{P}(B)-\mathbb{P}(A\cap B).
\]


\vspace*{1in}{}
\myquestion[5] Let $\mathcal{A}=\{(-\infty,x]\colon x\in\mathbb{R}\}$ and $\mathcal{B}=\{[x,\infty)\colon x\in\mathbb{R}\}$.
Show that $\mathcal{A}\subset\sigma(\mathcal{B})$.

\vspace*{1in}{}
\myquestion[5] Flip $4$ fair coins. Conditional on there being an even number of
heads, what is the probability that there are at least 2 heads?

\vspace*{1in}{}
\myquestion[5] Let $X$ be an r.v. and $F$ be its distribution function. Show that
$\mathbb{P}\{X<x\}=\lim_{y\uparrow x}F(y)$.

\newpage{}
\myquestion[5] $f:\mathbb{R}\rightarrow\mathbb{R}$ is \emph{upper semicontinuous
}if it can be written as the limit of a sequence $(f_{n})_{n\geq0}$
of nonincreasing (i.e., $f_{n}\geq f_{n+1}$) continuous functions.
Show that $f$ is Borel measurable.\\
\textbf{Hint}: it is sufficient to show that $f^{-1}((-\infty,a])$
is a Borel set, which you can do by using the fact that $f_{n}^{-1}((-\infty,a])$
is a Borel set for each $n$.

\vspace*{1in}{}
\myquestion[5] Let $X$ be a positive integer-valued r.v. Show that 
\[
\mathbb{P}\{X=n\}=\frac{1}{n(n+1)}\qquad\text{for }n\geq1
\]
defines a probability distribution.\textbf{ Hint}: $\frac{1}{n(n+1)}=\frac{1}{n}-\frac{1}{n+1}$.

\vspace*{1in}{}
\myquestion[5] Let $X$ be an r.v. Give an example of a function $f$ such that
$f\circ X$ is an r.v. that is independent of $X$.\textbf{ Hint}:
$f$ can be as simple as you want.

\vspace*{1in}{}
\myquestion[5] Let $X$ and $Y$ be nonnegative integer-valued r.v.s. Show that
\[
\mathbb{P}\{X\leq Y\}=\sum_{n=0}^{\infty}p_{n}\sum_{k=0}^{n}q_{k}
\]
where $p_{n}=\mathbb{P}\{Y=n\}$ and $q_{n}=\mathbb{P}\{X=n\}$.

\newpage{}
\myquestion[5] Let $X$ be an r.v. with MGF $M(\theta)=\frac{\lambda}{\lambda-\theta}$.
Then $\mathbb{E}[X]$ is equal to...
\begin{choices}
\mychoice $\frac{\lambda}{\left(\lambda-\theta\right)^{2}}$
\mychoice $\frac{1}{\lambda}$
\mychoice $\frac{2\lambda}{\left(\lambda-\theta\right)^{3}}$
\mychoice $\frac{2}{\lambda^{2}}$\medskip{}
\end{choices}
\myquestion[5] Show that 
\[
(\mathbb{E}[X])^{2}\leq\mathbb{E}[X^{2}]\qquad\text{for any r.v. }X.
\]
\textbf{Hint}: use the \emph{Cauchy-Schwarz inequality} ($\mathbb{E}[XY]\leq\sqrt{\mathbb{E}[X^{2}]\mathbb{E}[Y^{2}]}$).

\vspace*{1in}{}
\myquestion[5] Let $(X_{n})_{n}$ be a sequence of r.v.s such that $X_{n}\rightarrow\infty$
a.s. Show that the probability that $X_{n}\geq1$ for all but finitely
many $n$ is one. That is, show that 
\[
\mathbb{P}(\{\omega\mid\exists N\colon\forall n\geq N\colon X_{n}(\omega)\geq1\})=1.
\]
\textbf{Hint}: $X_{n}\rightarrow X$ a.s. means $\mathbb{P}(\{\omega\colon\lim_{n\rightarrow\infty}X_{n}(\omega)=X(\omega)\})=1$.

\vspace*{1in}{}
\myquestion[5] Let $(\Lambda_{n})_{n}$ be a sequence of events with $\mathbb{P}(\Lambda_{n})\rightarrow0$.
Let $X$ be an integrable random variable. Show that $\mathbb{E}[XI_{\Lambda_{n}}]\rightarrow0$.\textbf{
Hint}: $X$ is integrable means $\mathbb{E}[|X|]<\infty$.

\newpage{}
\myquestion[5] For square-integrable r.v.s $X$ and $Y$, show that $\sqrt{\mathbb{E}[(X+Y)^{2}]}\leq\sqrt{\mathbb{E}[X^{2}]}+\sqrt{\mathbb{E}[Y^{2}]}$.\\
\textbf{Hint}: expand $\mathbb{E}[(X+Y)^{2}]$ and use the Cauchy-Schwarz
inequality.

\vspace*{1in}{}
\myquestion[5] Let $(X_{n})_{n\geq1}$ be a sequence of r.v.s and $f\geq0$ be a
\uline{continuous} function such that
\[
\mathbb{E}\left[f(X_{n})\right]\leq\frac{1}{n}\qquad\text{for all }n\geq1.
\]
Suppose $X_{n}\rightarrow x\in\mathbb{R}$ in $L^{p}$ for some $p$.
Show that $f(x)=0$.\textbf{}\\
\textbf{Hint}: use the fact that $L^{p}$ convergence implies a.s.
convergence along a subsequence and apply \emph{Fatou's lemma}, which
tells us $\liminf_{n\rightarrow\infty}\mathbb{E}[f(X_{n})]\geq\mathbb{E}[\liminf_{n\rightarrow\infty}f(X_{n})]$.

\vspace*{1in}{}
\myquestion[5] Repeatedly flip a coin with probability $p$ of heads. Let $X_{n}$
be the number of heads seen after the $n$-th flip ($X_{0}=0$). Let
$\tau$ be the number of flips until you see $N\geq1$ heads. Then,
$\mathbb{E}[\tau]=\sum_{n\geq1}n\mu^{\intercal}\left(P^{n}-P^{n-1}\right)\nu$
where $\mu=\begin{pmatrix}1 & 0 & \cdots & 0\end{pmatrix}^{\intercal}$,
$\nu=\begin{pmatrix}0 & \cdots & 0 & 1\end{pmatrix}^{\intercal}$,
and $P\in\mathbb{R}^{(N+1)\times(N+1)}$. Fill in the nonzero entries
of $P$:
\[
P=\begin{pmatrix}\boxed{\phantom{p}} & \boxed{\phantom{p}}\\
 & \boxed{\phantom{p}} & \boxed{\phantom{p}}\\
 &  & \ddots & \ddots\\
 &  &  & \boxed{\phantom{p}} & \boxed{\phantom{p}}\\
 &  &  &  & \boxed{\phantom{p}}
\end{pmatrix}
\]
\medskip{}
\myquestion[5] Let $f:[0,1]^{d}\rightarrow\mathbb{R}$. To reduce the error $|\frac{1}{N}\sum_{n=1}^{N}f(z^{n})-\int_{[0,1]^{d}}f(\boldsymbol{x})d\boldsymbol{x}|$
in quasi-Monte Carlo, you should pick the sequence $(z^{n})_{n}$
such that...
\begin{choices}
\mychoice The variation of $f$ is as low as possible.
\mychoice The variation of $f$ is as high as possible.
\mychoice The star discrepancy of $P=\{z^{1},\ldots,z^{N}\}$ (for each $N$)
is as low as possible.
\mychoice The star discrepancy of $P=\{z^{1},\ldots,z^{N}\}$ (for each $N$)
is as high as possible.
\end{choices}

\newpage{}
\myquestion[5] You have a discrete stochastic process $(X_{n})_{n\in T}$ where
$T=\{0,1,\ldots,N\}$. You can transform this into a Markov process
by defining $(Y_{n})_{n\in T}$ by
\[
Y_{n}=(X_{0},\ldots,X_{n}).
\]
Suppose the state space of the $X$ process is $S$ with $|S|<\infty$.
How large is the state space of the $Y$ process?
\begin{choices}
\mychoice $\sum_{n=0}^{N}|S|^{n+1}=|S|\frac{|S|^{N+1}-1}{|S|-1}$
\mychoice $\prod_{n=0}^{N}|S|^{n+1}=|S|^{\frac{1}{2}(N+1)(N+2)}$
\mychoice $|S|^{N+1}$
\mychoice $(N+1)|S|$\medskip{}
\end{choices}
\myquestion[5] Consider the matrix $\left(\begin{smallmatrix}1/2&0&1/2\\2&0&0\\0&1&0\end{smallmatrix}\right)$.
This matrix is... (check \uline{all} that apply)
\begin{checkboxes}
\mychoice irreducible
\mychoice aperiodic
\mychoice primitive
\mychoice a transition matrix\medskip{}
\end{checkboxes}
\myquestion[5] Check \uline{all} true statements.
\begin{checkboxes}
\mychoice A discrete process is said to satisfy the \emph{Markov (a.k.a. memoryless)
property} if the conditional probability distribution of future states
(conditional on both past and present states) depend only upon the
present state.
\mychoice The \emph{strong Markov property} establishes a strong law of large
numbers for a discrete process that satisfies the Markov property.
\mychoice The \emph{strong Markov property} extends the Markov property to stopping
times.
\mychoice Any random variable which takes values in $[0,\infty]$ is a \emph{stopping
time}.\medskip{}
\end{checkboxes}
\myquestion[5] Fill in the blank. The PageRank algorithm represents a random surfer's
location in the web as a \_\_\_\_\_\_ Markov chain. This guarantees
that it has a limiting distribution.
\begin{choices}
\mychoice irreducible
\mychoice regular
\mychoice stationary
\mychoice nonnegative\medskip{}
\end{choices}
\end{myquestions}

\end{document}
