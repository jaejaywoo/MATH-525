%% LyX 2.2.3 created this file.  For more info, see http://www.lyx.org/.
%% Do not edit unless you really know what you are doing.
\documentclass[english,12pt]{article}
\usepackage[T1]{fontenc}
\usepackage[latin9]{inputenc}
\usepackage{amsmath}
\usepackage{amssymb}
\PassOptionsToPackage{normalem}{ulem}
\usepackage{ulem}

\makeatletter
%%%%%%%%%%%%%%%%%%%%%%%%%%%%%% User specified LaTeX commands.
\usepackage[margin=1in]{geometry}

\makeatother

\usepackage{babel}
\begin{document}

\title{Math 525: Assignment 2 Solutions}

\date{\date{}}
\maketitle
\begin{enumerate}
\item Note that
\begin{align*}
\mathbb{P}(A^{c}\cap B^{c}) & =\mathbb{P}(\left(A\cup B\right)^{c})\\
 & =1-\mathbb{P}(A\cup B)\\
 & =1-\left(\mathbb{P}(A)+\mathbb{P}(B)-\mathbb{P}(A\cap B)\right)\\
 & =\mathbb{P}(A^{c})-\mathbb{P}(B)+\mathbb{P}(A\cap B)\\
 & =\mathbb{P}(A^{c})-\mathbb{P}(B)+\mathbb{P}(A)\mathbb{P}(B)\\
 & =\mathbb{P}(A^{c})-\left(1-\mathbb{P}(A)\right)\mathbb{P}(B)\\
 & =\mathbb{P}(A^{c})-\mathbb{P}(A^{c})\mathbb{P}(B)\\
 & =\mathbb{P}(A^{c})\left(1-\mathbb{P}(B)\right)\\
 & =\mathbb{P}(A^{c})\mathbb{P}(B^{c}).
\end{align*}
Similarly, note that
\begin{align*}
\mathbb{P}(A\cap B^{c}) & =\mathbb{P}(A\setminus\left(A\cap B\right))\\
 & =\mathbb{P}(A)-\mathbb{P}(A\cap B)\\
 & =\mathbb{P}(A)-\mathbb{P}(A)\mathbb{P}(B)\\
 & =\mathbb{P}(A)\left(1-\mathbb{P}(B)\right)\\
 & =\mathbb{P}(A)\mathbb{P}(B^{c}).
\end{align*}
\item Use the definition of conditional probability:
\begin{align*}
\mathbb{P}(\text{at least one die is four}\mid\text{sum is seven}) & =\frac{\mathbb{P}(\text{at least one die is four and sum is seven})}{\mathbb{P}(\text{sum is seven})}\\
 & =\frac{2\mathbb{P}\left\{ (3,4)\right\} }{2\mathbb{P}\left\{ (1,6),(2,5),(3,4)\right\} }=\frac{1/36}{3/36}=\frac{1}{3}.
\end{align*}
\item ~
\begin{enumerate}
\item We need only check to make sure $\mathbb{T}$ is a probability measure.
First, note that
\[
0\leq\frac{\mathbb{P}(A\cap B)}{\mathbb{P}(B)}\leq\frac{\mathbb{P}(B)}{\mathbb{P}(B)}=1
\]
and hence $0\le\mathbb{T}(\cdot)\leq1$. Moreover, 
\[
\mathbb{T}(\emptyset)=\frac{\mathbb{P}(\emptyset\cap B)}{\mathbb{P}(B)}=\frac{0}{\mathbb{P}(B)}=0
\]
and
\[
\mathbb{T}(\Omega)=\frac{\mathbb{P}(\Omega\cap B)}{\mathbb{P}(B)}=\frac{\mathbb{P}(B)}{\mathbb{P}(B)}=1.
\]
Lastly, given disjoint sets $A_{1},\ldots,A_{n}\in\mathcal{F}$, note
that
\begin{multline*}
\mathbb{T}(A_{1}\cup\cdots\cup A_{n})=\frac{\mathbb{P}(\left(A_{1}\cup\cdots\cup A_{n}\right)\cap B)}{\mathbb{P}(B)}=\frac{\mathbb{P}(\left(A_{1}\cap B\right)\cup\cdots\cup\left(A_{n}\cap B\right))}{\mathbb{P}(B)}\\
=\frac{\mathbb{P}(A_{1}\cap B)+\cdots+\mathbb{P}(A_{n}\cap B)}{\mathbb{P}(B)}=\mathbb{T}(A_{n}).
\end{multline*}
\item Conditional probability is only defined if $\mathbb{P}(B)>0$. Any
reasonable definition for the case of $\mathbb{P}(B)=0$ is uninteresting.
\end{enumerate}
\item Let $s_{i}$ denote the $i$-th guest at the table, so that $s_{1},\ldots,s_{N}$
is the entire table. Guests $s_{i}$ and $s_{i+1}$ are adjacent for
any $1\leq i<N$, but so are guests $s_{1}$ and $s_{N}$ (the table
is circular). We use $B$ and $K$ to refer to Barbie and Ken.
\begin{enumerate}
\item Since there are $N!$ ways for the guests to sit, the answer is 
\[
N!-x=\boxed{N(N-3)(N-2)!}
\]
where $x$ is the answer in part (b).
\item We break the analysis up into cases:
\begin{enumerate}
\item If $s_{1}=B$ and $s_{N}=K$, Barbie and Ken are adjacent. The other
guests can be arranged in $(N-2)!$ ways in this case. We conclude
there are $(N-2)!$ arrangements in which Barbie sits at seat $s_{1}$
and Ken sits at seat $s_{N}$.
\item If $s_{i}=B$ and $s_{i+1}=K$, Barbie and Ken are adjacent. The other
guests can once again be arranged in $(N-2)!$ ways. Now, there are
$N-1$ ways we can pick $i$. We conclude there are $(N-1)(N-2)!=(N-1)!$
arrangements in which Barbie sits at seat $s_{i}$ and Ken sits at
seat $s_{i+1}$.
\item Since Barbie and Ken can switch seats and result in a valid arrangement,
we have a total of 
\[
2((N-2)!+(N-1)!)=\boxed{2N(N-2)!}
\]
 possible arrangements in which Barbie and Ken are adjacent.
\end{enumerate}
\item[\textbf{Note.}] There is another \uline{valid} way of interpreting the question.
We can consider the arrangement $s_{1},\ldots,s_{N}$ as equivalent
to the arrangement $s_{N},s_{1},s_{2},\ldots,s_{N-1}$ obtained by
having each guest move to the ``next'' seat. Alternatively, you
can think about this as ``rotating'' the table while keeping all
guests fixed. In this case, the answers to parts (a) and (b) are just
divided by $N$: 
\[
\boxed{(N-3)(N-2)!}\text{ and }\boxed{2(N-2)!}.
\]
\end{enumerate}
\item The base case $n=1$ is trivial. Suppose the binomial theorem holds
for some $n$. Then,
\begin{align*}
\left(a+b\right)^{n+1} & =\left(a+b\right)^{n}\left(a+b\right)\\
 & =\left(\sum_{k=0}^{n}\binom{n}{k}a^{k}b^{n-k}\right)\left(a+b\right)\\
 & =\sum_{k=0}^{n}\binom{n}{k}a^{k+1}b^{n-k}+\sum_{k=0}^{n}\binom{n}{k}a^{k}b^{n-(k-1)}\\
 & =\sum_{k=1}^{n}\binom{n}{k-1}a^{k}b^{n-(k-1)}+\sum_{k=0}^{n}\binom{n}{k}a^{k}b^{n-(k-1)}\\
 & =\sum_{k=1}^{n}\left(\binom{n}{k-1}+\binom{n}{k}\right)a^{k}b^{n-(k-1)}+\binom{n}{0}a^{0}b^{n+1}\\
 & =\sum_{k=1}^{n}\binom{n+1}{k}a^{k}b^{n-(k-1)}+\binom{n}{0}a^{0}b^{n+1}\\
 & =\sum_{k=1}^{n}\binom{n+1}{k}a^{k}b^{n+1-k}+\binom{n}{0}a^{0}b^{n+1}\\
 & =\sum_{k=0}^{n}\binom{n+1}{k}a^{k}b^{n+1-k}
\end{align*}
where we used the fact that
\begin{align*}
\binom{n}{k-1}+\binom{n}{k} & =\frac{n!}{\left(k-1\right)!\left(n-k+1\right)!}+\frac{n!}{k!\left(n-k\right)!}\\
 & =\frac{n!k}{k!\left(n-k+1\right)!}+\frac{n!\left(n-k+1\right)}{k!\left(n-k\right)!}\\
 & =\frac{n!\left(k+n-k+1\right)}{k!\left(n-k+1\right)!}\\
 & =\frac{n!\left(n+1\right)}{k!\left(n+1-k\right)!}\\
 & =\binom{n+1}{k}.
\end{align*}
\item ~
\begin{enumerate}
\item ~
\begin{align*}
x\in f^{-1}(H\cup J) & \iff f(x)\in H\cup J\\
 & \iff f(x)\in H\text{ or }f(x)\in J\\
 & \iff x\in f^{-1}(H)\text{ or }x\in f^{-1}(J)\\
 & \iff x\in f^{-1}(H)\cup f^{-1}(J).
\end{align*}
\item ~
\begin{align*}
x\in f^{-1}(H^{c}) & \iff f(x)\in H^{c}\\
 & \iff f(x)\notin H\\
 & \iff x\notin f^{-1}(H)\\
 & \iff x\in\left(f^{-1}(H)\right)^{c}.
\end{align*}
\item By parts (a) and (b),
\begin{align*}
f^{-1}(H\cap J) & =\left(\left(f^{-1}(H\cap J)\right)^{c}\right)^{c}\\
 & =\left(f^{-1}(H^{c}\cup J^{c})\right)^{c}\\
 & =\left(f^{-1}(H^{c})\cup f^{-1}(J^{c})\right)^{c}\\
 & =\left(f^{-1}(H^{c})\right)^{c}\cap\left(f^{-1}(J^{c})\right)^{c}\\
 & =f^{-1}(H)\cap f^{-1}(J).
\end{align*}
\end{enumerate}
\item To show that $X=\sup_{n}X_{n}$, it is enough to check that sets of
the form $\{X\leq x\}$ are in the underlying $\sigma$-algebra, call
it $\mathcal{F}$. Note that
\[
\left\{ X\leq x\right\} =\left\{ \sup_{n}X_{n}\leq x\right\} =\bigcap_{n\geq1}\left\{ X_{n}\leq x\right\} .
\]
Since each $X_{n}$ is a random variable, we know that $\{X_{n}\leq x\}\in\mathcal{F}$.
Since $\mathcal{F}$ is closed under countable intersections, the
desired result follows.
\end{enumerate}

\end{document}
