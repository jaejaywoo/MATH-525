%% LyX 2.2.3 created this file.  For more info, see http://www.lyx.org/.
%% Do not edit unless you really know what you are doing.
\documentclass[english,12pt]{article}
\usepackage[T1]{fontenc}
\usepackage[latin9]{inputenc}
\usepackage{amssymb}
\usepackage{esint}
\PassOptionsToPackage{normalem}{ulem}
\usepackage{ulem}

\makeatletter
%%%%%%%%%%%%%%%%%%%%%%%%%%%%%% User specified LaTeX commands.
\usepackage[margin=1in]{geometry}

\makeatother

\usepackage{babel}
\begin{document}

\title{Math 525: Assignment 4}

\date{\date{}}
\maketitle
\begin{enumerate}
\item (Moment generating functions) Let $X$ be a random variable with distribution
function
\[
F(x)=\frac{1}{\sqrt{2\pi}}\int_{-\infty}^{x}e^{-u^{2}/2}du
\]
(we call $X$ a \emph{standard normal random variable}; you may have
seen it before). Compute the third raw moment of $X$ \uline{using
the moment generating function}. \\
\textbf{Hint}: If $X$ is an absolutely continuous distribution function
$F$ with density $f$ (i.e., $F(x)=\int_{-\infty}^{x}f(y)dy$) and
if $g\geq0$ is a Borel measurable function, then
\[
\mathbb{E}\left[g(X)\right]=\int_{-\infty}^{\infty}g(u)f(u)du.
\]
\item (Expectations) Let $X$ be a nonnegative integrable random variable.
\begin{enumerate}
\item Show that $X=0$ a.e. implies $\mathbb{E}X=0$.
\item Show that $\mathbb{E}X=0$ implies $X=0$ a.e.\textbf{ Hint}: consider
sets of the form $A_{n}=\{X\geq1/n\}$ where $n$ is a positive integer.
\item Let $E$ be a set of probability zero (i.e., $\mathbb{P}(E)=0$).
Use part (a) to show that $\mathbb{E}[XI_{E}]=0$ where $I_{E}$ is
the indicator random variable on $E$.
\end{enumerate}
\item (Expectations) Let $X$ be a nonnegative random variable with distribution
function $F$.
\begin{enumerate}
\item Show that, for any real number $x\geq0$,
\[
x=\int_{0}^{\infty}I_{[0,x)}(y)dy
\]
where $I_{[0,x)}(y)=1$ if $0\leq y<x$ and $I_{[0,x)}(y)=0$ otherwise.
Use this to derive
\[
\mathbb{E}X=\mathbb{E}\left[\int_{0}^{\infty}I_{[0,X)}(y)dy\right].
\]
\item (Optional) Show, using the results of part (a), that
\[
\mathbb{E}X=\int_{0}^{\infty}\mathbb{E}\left[I_{[0,X)}(y)\right]dy.
\]
\item Show that $\mathbb{E}[I_{[0,X)}(y)]=\mathbb{P}(X>y).$
\item Combine your findings in parts (b) and (c) to conclude
\[
\mathbb{E}X=\int_{0}^{\infty}\left(1-F(y)\right)dy.
\]
\end{enumerate}
\end{enumerate}

\end{document}
