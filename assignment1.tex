%% LyX 2.2.1 created this file.  For more info, see http://www.lyx.org/.
%% Do not edit unless you really know what you are doing.
\documentclass[english,12pt]{article}
\usepackage[T1]{fontenc}
\usepackage[latin9]{inputenc}
\usepackage{amsmath}
\usepackage{amssymb}

\makeatletter
%%%%%%%%%%%%%%%%%%%%%%%%%%%%%% User specified LaTeX commands.
\usepackage[margin=1in]{geometry}

\makeatother

\usepackage{babel}
\begin{document}

\title{Math 525: Assignment 1}

\date{\date{}}
\maketitle
\begin{enumerate}
\item Sometimes it is useful to start with a small set of events $\mathcal{G}$
that is not necessarily a $\sigma$-algebra and ``generate'' a $\sigma$-algebra
from it. The $\sigma$-algebra generated from $\mathcal{G}$ is
\[
\sigma(\mathcal{G})=\bigcap_{\substack{\substack{\mathcal{F}\text{ is a }\sigma}
\text{-algebra on }\Omega\\
\mathcal{G}\subset\mathcal{F}
}
}\mathcal{F}.
\]
That is, $\sigma(\mathcal{G})$ is the intersection of all $\sigma$-algebras
containing $\mathcal{G}$. Show that...
\begin{enumerate}
\item $\sigma(\mathcal{G})$ is a $\sigma$-algebra.
\item $\sigma(\mathcal{G})\subset\sigma(\mathcal{G}^{\prime})$ whenever
$\mathcal{G}\subset\mathcal{G}^{\prime}$.
\item If $\mathcal{F}$ is a $\sigma$-algebra, then $\sigma(\mathcal{F})=\mathcal{F}$.
\item If $\mathcal{F}$ is a $\sigma$-algebra and $\mathcal{G}\subset\mathcal{F}$,
then $\sigma(\mathcal{G})\subset\mathcal{F}$.
\end{enumerate}
\item Show that the following two are algebras but not $\sigma$-algebras:
\begin{enumerate}
\item All finite subsets of $\mathbb{R}$ together with their complements.
\item All finite unions of intervals in $\mathbb{R}$ of the form $(a,b]$,
$(-\infty,a]$, and $(b,\infty)$.
\end{enumerate}
\item Prove the principle of inclusion-exclusion. That is, show that if
$A_{1},\ldots,A_{n}\in\mathcal{F}$, then
\begin{multline*}
\mathbb{P}(A_{1}\cup\cdots\cup A_{n})\\
=\sum_{i}\mathbb{P}(A_{i})-\sum_{i<j}\mathbb{P}(A_{i}\cap A_{j})+\sum_{i<j<k}\mathbb{P}(A_{i}\cap A_{j}\cap A_{k})-\cdots+(-1)^{n+1}\mathbb{P}(A_{1}\cap\cdots\cap A_{n}).
\end{multline*}
Note that when we write $\sum_{i<j}$, we mean $\sum_{i,j\in\{(i,j)\colon i<j\}}$
(and similarly for the sums involving more indices).
\end{enumerate}

\end{document}
