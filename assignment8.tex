%% LyX 2.2.1 created this file.  For more info, see http://www.lyx.org/.
%% Do not edit unless you really know what you are doing.
\documentclass[english,12pt]{article}
\usepackage[T1]{fontenc}
\usepackage[latin9]{inputenc}
\usepackage{units}
\usepackage{amstext}
\usepackage{amsthm}
\usepackage{amssymb}

\makeatletter
%%%%%%%%%%%%%%%%%%%%%%%%%%%%%% Textclass specific LaTeX commands.
 \theoremstyle{definition}
 \newtheorem*{defn*}{\protect\definitionname}
  \theoremstyle{plain}
  \newtheorem*{thm*}{\protect\theoremname}

%%%%%%%%%%%%%%%%%%%%%%%%%%%%%% User specified LaTeX commands.
\usepackage[margin=1in]{geometry}

\makeatother

\usepackage{babel}
  \providecommand{\definitionname}{Definition}
  \providecommand{\theoremname}{Theorem}

\begin{document}

\title{Math 525: Assignment 8}

\date{\date{}}

\maketitle
To answer the first question, you will need a theorem from linear
algebra. Recall that a matrix/vector is said to be nonnegative/positive
if all of its entries are nonnegative/positive.
\begin{defn*}
Let $A$ be a nonnegative matrix. We say $A$ is \emph{primitive}
if $A^{m}$ is a positive matrix for some positive integer $m$.
\end{defn*}
\begin{thm*}[Perron-Frobenius for primitive matrices]
Let $A$ be a primitive matrix. Then,
\begin{itemize}
\item $\lambda_{1}=\rho(A)$ is an eigenvalue of $A$.
\item All other eigenvalues $\lambda_{j}$ are smaller in magnitude than
$\lambda_{1}$ (i.e., $|\lambda_{j}|<\lambda_{1}$).
\item $\{x\colon Ax=\lambda_{1}x\}=\text{span}(v)$ for some positive vector
$v$.
\end{itemize}
\end{thm*}
\noindent\hrulefill
\begin{enumerate}
\item A Markov chain whose transition matrix $P$ is primitive is called
\emph{regular}.\emph{ }Let
\begin{itemize}
\item $\mu=(\mu_{1},\ldots,\mu_{m})^{\intercal}$ be a column vector with
$\mu^{\intercal}e=1$ where $e=(1,\ldots,1)^{\intercal}$,
\item $P$ be the transition matrix of a regular Markov chain, and
\item suppose that $P^{\intercal}$ has a linearly independent set of eigenvectors
$\{v_{1},\ldots,v_{m}\}$ with corresponding eigenvalues $\{\lambda_{1},\ldots,\lambda_{m}\}$
in descending order of magnitude: 
\[
|\lambda_{1}|>|\lambda_{2}|\geq\cdots\geq|\lambda_{m}|.
\]
\end{itemize}
\begin{enumerate}
\item Show that if $\mu^{\intercal}e=1$, then $\mu^{\intercal}Pe=1$.
\item Let $\mu=c_{1}v_{1}+\cdots+c_{m}v_{m}$ be an eigendecomposition of
$\mu$. By the Perron-Frobenius theorem, we can, without loss of generality,
pick $v_{1}$ to be a positive vector. Show that for any positive
integer $k$,
\[
\mu^{\intercal}P^{k}=c_{1}\lambda_{1}^{k}v_{1}^{\intercal}+\cdots+c_{m}\lambda_{m}^{k}v_{m}^{\intercal}.
\]
\item Show that $\lim_{n\rightarrow\infty}\mu^{\intercal}P^{n}=c_{1}v_{1}^{\intercal}$\textbf{
}(\textbf{Hint}: Proposition 1.13 of Lecture 16).
\item Show that $(c_{1}v_{1}^{\intercal})P=c_{1}v_{1}^{\intercal}$.
\item Show that $(c_{1}v_{1}^{\intercal})e=1$ (\textbf{Hint}: part (a))
and $c_{1}>0$.
\item Part (e) shows that $c_{1}v_{1}^{\intercal}$ is a distribution vector
(i.e., it is nonnegative and its entries add up to one). With this
in mind, what is the significance of
\begin{enumerate}
\item Part (c)? (\textbf{Hint}: if $\mathbb{P}(X_{0}=i)=\mu_{i}$, then
$(\mu^{\intercal}P^{n})_{i}=\mathbb{P}(X_{n}=i)$).
\item Part (d)?
\end{enumerate}
\end{enumerate}
\end{enumerate}
\clearpage{}
\begin{enumerate}
\item[2.] A coin with probability $p$ of heads is flipped repeatedly. $X_{n}$
is the result of the $n$-th coin flip ($n=1$ is the first coin flip).
Let
\[
\tau=\inf\left\{ n>1\colon(X_{n-1},X_{n})=(H,T)\right\} ,
\]
corresponding to the first time at which we see a heads ($H$) followed
by a tails ($T$).
\begin{enumerate}
\item Give an expression for $\mathbb{P}(\tau\leq n)$ (\textbf{Hint}: try
to think of a transition matrix $P$ such that $\mathbb{P}(\tau\leq n)=\mu^{\intercal}P^{n}\nu$
for appropriately chosen column vectors $\mu$ and $\nu$).
\item Give an expression for $\mathbb{P}(\tau=n)$\textbf{ }(\textbf{Hint}:
use your results from part (a)).
\item (Optional) Let $p=\nicefrac{1}{3}$ and compute $\mathbb{P}(\tau\leq10)$
with numerical software such as MATLAB.
\end{enumerate}
\end{enumerate}

\end{document}
