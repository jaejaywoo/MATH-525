%% LyX 2.2.1 created this file.  For more info, see http://www.lyx.org/.
%% Do not edit unless you really know what you are doing.
\documentclass[english,12pt]{article}
\usepackage[T1]{fontenc}
\usepackage[latin9]{inputenc}
\usepackage{mathtools}
\usepackage{amsmath}
\usepackage{amsthm}
\usepackage{amssymb}
\PassOptionsToPackage{normalem}{ulem}
\usepackage{ulem}

\makeatletter
%%%%%%%%%%%%%%%%%%%%%%%%%%%%%% Textclass specific LaTeX commands.
\theoremstyle{plain}
\newtheorem{thm}{\protect\theoremname}[section]
\theoremstyle{definition}
\newtheorem{defn}[thm]{\protect\definitionname}
\theoremstyle{remark}
\newtheorem{rem}[thm]{\protect\remarkname}
\theoremstyle{plain}
\newtheorem{prop}[thm]{\protect\propositionname}
\ifx\proof\undefined
\newenvironment{proof}[1][\protect\proofname]{\par
\normalfont\topsep6\p@\@plus6\p@\relax
\trivlist
\itemindent\parindent
\item[\hskip\labelsep\scshape #1]\ignorespaces
}{%
\endtrivlist\@endpefalse
}
\providecommand{\proofname}{Proof}
\fi
\theoremstyle{plain}
\newtheorem{cor}[thm]{\protect\corollaryname}
\theoremstyle{definition}
\newtheorem{example}[thm]{\protect\examplename}

%%%%%%%%%%%%%%%%%%%%%%%%%%%%%% User specified LaTeX commands.
\usepackage[margin=1in]{geometry}

\makeatother

\usepackage{babel}
\providecommand{\corollaryname}{Corollary}
\providecommand{\definitionname}{Definition}
\providecommand{\examplename}{Example}
\providecommand{\propositionname}{Proposition}
\providecommand{\remarkname}{Remark}
\providecommand{\theoremname}{Theorem}

\begin{document}

\title{Math 525: Lecture 23}

\date{April 10, 2018}
\maketitle

\section{Joint distribution}

In previous lectures, we have dealt with only one-dimensional distribution
functions (i.e., $F(x)=\mathbb{P}\{X\leq x\}$). In this lecture,
we extend our findings to higher dimensions.
\begin{defn}
The \emph{joint distribution} of two random variables $X$ and $Y$
(defined on the same probability space) is the function $F:\mathbb{R}\times\mathbb{R}\rightarrow[0,1]$
defined by
\[
F(x,y)=\mathbb{P}\left\{ X\leq x,Y\leq y\right\} .
\]
If we wish to be explicit, we may write $F_{XY}$.
\end{defn}
\begin{rem}
Similarly, we can define $F(x_{1},\ldots,x_{n})=\mathbb{P}\{X_{1}\leq x_{1},\ldots,X_{n}\leq x_{n}\}$
as the joint distribution of $n$ random variables. Since this extension
is trivial, we work solely with the two dimensional case for ease
of notation.
\end{rem}
 Recall that distribution functions satisfy some distinctive properties.
Let's now extend those to joint distribution functions.
\begin{prop}
Let $y$ be arbitrary. Then,
\begin{enumerate}
\item $x\mapsto F(x,y)$ is nondecreasing.
\item $x\mapsto F(x,y)$ is right-continuous.
\item $\lim_{x\rightarrow-\infty}F(x,y)=0$.
\item $\lim_{x\rightarrow\infty}F(x,y)=\mathbb{P}\{Y\leq y\}$.
\end{enumerate}
\end{prop}
\begin{proof}
Let $B^{y}=\{Y\leq y\}$ and for each $x$, let $A_{x}=\{X\leq x\}$.
\begin{enumerate}
\item Since $A_{x_{1}}\subset A_{x_{2}}$ whenever $x_{1}\leq x_{2}$, then
$A_{x_{1}}\cap B^{y}\subset A_{x_{2}}\cap B^{y}$ and hence $F(x_{1},y)=\mathbb{P}(A_{x_{1}}\cap B^{y})\leq\mathbb{P}(A_{x_{2}}\cap B^{y})=F(x_{2},y)$.
\item Let $x$ be arbitrary and $(x_{n})_{n\geq1}$ be a sequence with $x_{n}\downarrow x$.
Then, 
\[
A_{x_{1}}\cap B^{y}\supset A_{x_{2}}\cap B^{y}\supset\cdots
\]
 By continuity of measure,
\begin{multline*}
\lim_{n\rightarrow\infty}F(x_{n},y)=\lim_{n\rightarrow\infty}\mathbb{P}(A_{x_{n}}\cap B^{y})=\mathbb{P}\left(\bigcap_{n\geq1}\left(A_{x_{n}}\cap B^{y}\right)\right)\\
=\mathbb{P}\left(\left(\bigcap_{n\geq1}A_{x_{n}}\right)\cap B^{y}\right)=\mathbb{P}\left(A_{x}\cap B^{y}\right)=F(x,y).
\end{multline*}
\item Note that $A_{-1}\supset A_{-2}\supset\cdots$ and hence by continuity
of measure,
\[
\lim_{n\rightarrow\infty}F(-n,y)=\lim_{n\rightarrow\infty}\mathbb{P}(A_{-n}\cap B^{y})=\mathbb{P}(\emptyset)=0.
\]
\item Note that $A_{1}\subset A_{2}\subset\cdots$ and hence by continuity
of measure,
\[
\lim_{n\rightarrow\infty}F(n,y)=\lim_{n\rightarrow\infty}\mathbb{P}(A_{n}\cap B^{y})=\mathbb{P}(B^{y}).\qedhere
\]
\end{enumerate}
\end{proof}
Note that the last point tells us that we can retrieve the distribution
function of $Y$ by taking limits in $x$. That is, $F(\infty,y)\equiv\lim_{x\rightarrow\infty}F(x,y)=F_{Y}(y)$,
where $F_{Y}(y)=\mathbb{P}\{Y\leq y\}$ defines the distribution function
of $Y$. We call $F(\infty,y)$ the \emph{marginal distribution }of
$Y$. The act of taking limits in $x$ is called \emph{marginalization
}(i.e., ``summing out'' the variable you are not interested in).
\begin{cor}
Let $F_{Y}$ be the distribution function of $Y$. If $F_{Y}\neq0$,
then, $x\mapsto F(x,y)/F_{Y}(y)$ is a distribution function.
\end{cor}
Indeed, we can go a little further in our claim:
\[
\frac{F(x,y)}{F_{Y}(y)}=\frac{\mathbb{P}\left\{ X\leq x,Y\leq y\right\} }{\mathbb{P}\{Y\leq y\}}=\mathbb{P}(X\leq x\mid Y\leq y).
\]

\begin{proof}
We proved that $x\mapsto F(x,y)/F_{Y}(y)$ is nondecreasing, right-continuous,
its limit as $x\rightarrow-\infty$ is zero and its limit as $x\rightarrow\infty$
is
\[
\lim_{x\rightarrow\infty}\frac{F(x,y)}{F_{Y}(y)}=\frac{F_{Y}(y)}{F_{Y}(y)}=1.\qedhere
\]
\end{proof}
Using the same ideas as above (involving continuity of measure) we
can prove a few more facts about joint distribution functions. We
state them without proof.
\begin{prop}
~
\begin{enumerate}
\item $(x,y)\mapsto F(x,y)$ is nondecreasing with respect to the element-wise
partial order (i.e., $F(x_{1},y_{1})\leq F(x_{2},y_{2})$ whenever
$x_{1}\leq x_{2}$ and $y_{1}\leq y_{2}$).
\item $\lim_{x,y\rightarrow\infty}F(x,y)=1$.
\end{enumerate}
\end{prop}
As usual, we can define $F(x-,y)=\lim_{t\uparrow x}F(t,y)$, $F(x,y-)=\lim_{s\uparrow y}F(x,s)$,
and $F(x-,y-)=\lim_{t\uparrow x,s\uparrow y}F(t,s)$. We use the notation
$x\uparrow y$ to mean that $x$ converges to $y$ \emph{strictly}
from below (i.e., $x<y$ and $x\rightarrow y$).\footnote{Walsh uses the symbol $\upuparrows$ for this.}
\begin{prop}
Let $x_{1}<x_{2}$ and $y_{1}<y_{2}$. Then,
\begin{enumerate}
\item $\mathbb{P}\{X\leq x_{2},Y<y_{2}\}=F(x_{2},y_{2}-)$.
\item $\mathbb{P}\left\{ X<x_{2},Y\leq y_{2}\right\} =F(x_{2}-,y_{2})$.
\item $\mathbb{P}\left\{ X<x_{2},Y<y_{2}\right\} =F(x_{2}-,y_{2}-)$.
\item $\mathbb{P}\{x_{1}<X\leq x_{2},y_{1}<Y\leq y_{2}\}=F(x_{2},y_{2})-F(x_{1},y_{2})-F(x_{2},y_{1})+F(x_{1},y_{1})$.
\item $\mathbb{P}\{X=x_{1},Y=y_{1}\}=F(x_{1},y_{1})-F(x_{1}-,y_{1})-F(x_{1},y_{1}-)+F(x_{1}-,y_{1}-)$.
\end{enumerate}
\end{prop}
\begin{defn}
If $X$ and $Y$ are discrete random variables, their \emph{joint
probability mass function} $p:\mathbb{R}\times\mathbb{R}\rightarrow[0,1]$
(written $p_{XY}$ if we wish to be explicit) is defined by
\[
p_{XY}(x,y)=\mathbb{P}\left\{ X=x,Y=y\right\} .
\]
\end{defn}
In this case, the corresponding joint distribution function is given
by 
\[
F_{XY}(x,y)=\sum_{u\leq x,v\leq y}p_{XY}(u,v).
\]
Note that this sum is well defined since even though the set $A=\{(u,v)\colon u\leq x,v\leq y\}$
is uncountable, there are only countably many elements $(u,v)\in A$
such that $p_{XY}(u,v)$ is nonzero.

\section{Absolutely continuous distributions}
\begin{defn}
We say $F$ is \emph{absolutely continuous} if it can be written
\[
F(x,y)=\int_{-\infty}^{y}\int_{-\infty}^{x}f(u,v)dudv
\]
for some integrable function $f$. We call $f$ the joint density
of $X$ and $Y$ (sometimes written $f_{XY}$ to be explicit).
\end{defn}
From the definition,
\[
\mathbb{P}\left\{ x_{1}<X\leq x_{2},y_{1}<Y\leq y_{2}\right\} =\int_{x_{1}}^{x_{2}}\int_{y_{1}}^{y_{2}}f(x,y)dxdy.
\]
More generally, for any Borel measurable set $A\subset\mathbb{R}^{2}$,
\[
\mathbb{P}((X,Y)\in A)=\int\int_{A}f(x,y)dxdy.
\]
Moreover, by the fundamental theorem of calculus, if $f$ is continuous,
then
\[
f(x,y)=\frac{\partial^{2}}{\partial x\partial y}F(x,y).
\]
As with marginal distributions, we can also define \emph{marginal
densities}:
\begin{prop}
Suppose $X$ and $Y$ have a joint density $f_{XY}$. Then, $Y$ has
a density $f_{Y}$ satisfying 
\[
f_{Y}(y)=\int_{-\infty}^{\infty}f_{XY}(x,y)dx.
\]
\end{prop}
\begin{proof}
We obtain the desired result by showing that $F_{Y}$ is absolutely
continuous with integrand $f_{Y}$:
\[
F_{Y}(y)=\mathbb{P}\left\{ -\infty<X<\infty,Y\leq y\right\} =\int_{-\infty}^{y}\int_{-\infty}^{\infty}f_{XY}(u,v)dudv=\int_{-\infty}^{y}f_{Y}(v)dv.\qedhere
\]
\end{proof}
The above shows that if $X$ and $Y$ have a joint density, then $Y$
(and also $X$) also has a density. The converse is not true in general:
\begin{example}
Let $X\sim U[0,1]$ and $Y=X$ ($X$ and $Y$ are \uline{not} independent).
Of course, both of these variables have the density $f_{X}=f_{Y}=1$.
If they had a joint density $f_{XY}$, its probability mass would
have to be concentrated on the diagonal $\{(x,y)\colon x=y\}$. However,
the area of this diagonal is zero, and hence the integral $\int_{0}^{1}\int_{0}^{1}f_{XY}(x,y)dxdy$
would also be zero, a contradiction.
\end{example}
\begin{prop}
Suppose $X$ and $Y$ have a continuous joint density $f_{XY}$. Then,
they are independent if and only if $f_{XY}(x,y)=f_{X}(x)f_{Y}(y)$
at all points $x$ and $y$.
\end{prop}
Continuity is not actually required in the above if we are willing
to replace ``at all'' by ``at almost all'' since absolutely continuous
functions are differentiable almost everywhere (those of you who have
taken/are taking measure theory might have seen this).
\begin{proof}
Suppose $X$ and $Y$ are independent, so that $F_{XY}(x,y)=F_{X}(x)F_{Y}(y)$
for all $x$ and $y$. Then, for all $x$ and $y$,
\begin{multline*}
f_{XY}(x,y)=\frac{\partial^{2}}{\partial x\partial y}F_{XY}(x,y)=\frac{\partial^{2}}{\partial x\partial y}\left[F_{X}(x)F_{Y}(y)\right]=\left(\frac{\partial}{\partial x}F_{X}(x)\right)\left(\frac{\partial}{\partial y}F_{Y}(y)\right)\\
=f_{X}(x)f_{Y}(y).
\end{multline*}
Suppose $f_{XY}(x,y)=f_{X}(x)f_{Y}(y)$ for all $x$ and $y$. Then,
for all $x$ and $y$
\begin{multline*}
F_{XY}(x,y)=\int_{-\infty}^{y}\int_{-\infty}^{x}f_{XY}(u,v)dudv=\int_{-\infty}^{y}\int_{-\infty}^{x}f_{X}(u)f_{Y}(v)dudv\\
=\left(\int_{-\infty}^{x}f_{X}(u)du\right)\left(\int_{-\infty}^{y}f_{Y}(v)dv\right)=F_{X}(x)F_{Y}(y).\qedhere
\end{multline*}
\end{proof}

\section{Covariances and correlations}

So far, we know that two random variables can either be independent
or not. This is not particularly useful when we want to make statistical
inferences.

For example, if $X$ is someone's muscle mass and $Y$ is the average
amount of protein they eat per day, we certainly do not expect $X$
and $Y$ to be independent. We expect that $X$ to be large if $Y$
is large. That is, we expect $X$ and $Y$ to be ``positively correlated''.

Before we define correlation, let's define the covariance:
\begin{defn}
Let $X$ and $Y$ be square integrable random variables. The \emph{covariance
}of $X$ and $Y$ is
\[
\operatorname{Cov}(X,Y)=\mathbb{E}\left[\left(X-\mathbb{E}X\right)\left(Y-\mathbb{E}Y\right)\right].
\]
\end{defn}
Note that the covariance above is well-defined since by Cauchy-Schwarz,
\begin{equation}
\left|\operatorname{Cov}(X,Y)\right|\leq\sqrt{\mathbb{E}\left[\left(X-\mathbb{E}X\right)^{2}\right]\mathbb{E}\left[\left(Y-\mathbb{E}Y\right)^{2}\right]}=\sqrt{\operatorname{Var}(X)\operatorname{Var}(Y)}<\infty.\label{eq:cauchy}
\end{equation}
Note also that $\operatorname{Cov}(X,X)=\operatorname{Var}(X)$. The
correlation is just the normalized covariance:
\begin{defn}
Let $X$ and $Y$ be square integrable random variables each having
nonzero variance. The \emph{correlation} of $X$ and $Y$ is
\[
\operatorname{Cor}(X,Y)=\frac{\operatorname{Cov}(X,Y)}{\sqrt{\operatorname{Var}(X)}\sqrt{\operatorname{Var}(Y)}}.
\]
\end{defn}
Note that in the above, the correlation is undefined when either $X$
and $Y$ have variance zero. However, this is not a particularly interesting
case, since a random variable with variance zero (i.e., a random variable
which ``does not vary'') is actually deterministic:
\begin{prop}
Let $X$ be a square integrable random variable with $\operatorname{Var}(X)=0$.
Then, $X=\mathbb{E}X$ a.s.
\end{prop}
\begin{proof}
This follows directly from the fact that $\operatorname{Var}(X)=\mathbb{E}[(X-\mathbb{E}[X])^{2}]=0$
implies that $(X-\mathbb{E}[X])^{2}=0$ a.s.
\end{proof}
\begin{prop}
If $X$ and $Y$ are square-integrable, then
\begin{enumerate}
\item $-1\leq\operatorname{Cor}(X,Y)\leq1$.
\item $\operatorname{Cov}(X,Y)=\mathbb{E}[XY]-\mathbb{E}[X]\mathbb{E}[Y]$.
\item $\operatorname{Cor}(X,Y)=1\iff Y=aX+b$ a.s. for some constants $a>0$
and $b$.
\item $\operatorname{Cor}(X,Y)=-1\iff Y=aX+b$ a.s. for some constants $a<0$
and $b$.
\item The correlation is the covariance of the normalized random variables:
\[
\operatorname{Cor}(X,Y)=\operatorname{Cov}\left(\frac{X}{\sqrt{\operatorname{Var}(X)}},\frac{Y}{\sqrt{\operatorname{Var}(Y)}}\right).
\]
\item If $X$ and $Y$ are independent, then $\operatorname{Cor}(X,Y)=\operatorname{Cov}(X,Y)=0$.
\end{enumerate}
\end{prop}
\begin{proof}
~
\begin{enumerate}
\item By (\ref{eq:cauchy}).
\item This follows from
\begin{equation}
\operatorname{Cov}(X,Y)=\mathbb{E}\left[XY-X\mathbb{E}Y-Y\mathbb{E}X+\mathbb{E}X\mathbb{E}Y\right]=\mathbb{E}\left[XY\right]-\mathbb{E}X\mathbb{E}Y.\label{eq:cov}
\end{equation}
\item Suppose $Y=aX+b$ for some $a>0$ and $b$. Then,
\begin{multline*}
\operatorname{Cov}(X,Y)=\operatorname{Cov}(X,aX+b)=\mathbb{E}\left[X\left(aX+b\right)\right]-\mathbb{E}X\mathbb{E}\left[aX+b\right]\\
=a\mathbb{E}\left[X^{2}\right]+b\mathbb{E}\left[X\right]-\mathbb{E}X\left(a\mathbb{E}\left[X\right]+b\right)=a\left(\mathbb{E}\left[X^{2}\right]-\left(\mathbb{E}X\right)^{2}\right)=a\operatorname{Var}(X).
\end{multline*}
Moreover, $\operatorname{Var}(Y)=\operatorname{Var}(aX+b)=a^{2}\operatorname{Var}(X)$.
Therefore,
\[
\operatorname{Cor}(X,Y)=\frac{\operatorname{Cov}(X,Y)}{\sqrt{a^{2}\operatorname{Var}(X)}\sqrt{\operatorname{Var}(X)}}=\frac{a\operatorname{Var}(X)}{a\operatorname{Var}(X)}=1.
\]
Now, suppose $\operatorname{Cor}(X,Y)=1$. This means that (\ref{eq:cauchy})
holds with equality, which only occurs if
\[
Y-\mathbb{E}Y=a\left(X-\mathbb{E}X\right)\qquad\text{a.s.}
\]
for some constant $a$ (we proved this in an early lecture). Therefore,
\[
Y=aX\underbrace{-a\mathbb{E}X+\mathbb{E}Y}_{b}\qquad\text{a.s.}
\]
\item Similar to the above.
\item This follows by two applications of (\ref{eq:cov}):
\begin{multline*}
\operatorname{Cov}\left(\frac{X}{\sqrt{\operatorname{Var}(X)}},\frac{Y}{\sqrt{\operatorname{Var}(Y)}}\right)=\mathbb{E}\left[\frac{XY}{\sqrt{\operatorname{Var}(X)}\sqrt{\operatorname{Var}(Y)}}\right]-\frac{\mathbb{E}X\mathbb{E}Y}{\sqrt{\operatorname{Var}(X)}\sqrt{\operatorname{Var}(Y)}}.\\
=\frac{\mathbb{E}\left[XY\right]-\mathbb{E}X\mathbb{E}Y}{\sqrt{\operatorname{Var}(X)}\sqrt{\operatorname{Var}(Y)}}=\frac{\operatorname{Cov}(X,Y)}{\sqrt{\operatorname{Var}(X)}\sqrt{\operatorname{Var}(Y)}}=\operatorname{Cor}(X,Y).
\end{multline*}
\item By (\ref{eq:cov}).\qedhere
\end{enumerate}
\end{proof}
\begin{defn}
If $\operatorname{Cov}(X,Y)=0$, we say $X$ and $Y$ are \emph{orthogonal}.
\end{defn}
While independence implies orthogonality, the converse is not true
in general.
\end{document}
