%% LyX 2.2.1 created this file.  For more info, see http://www.lyx.org/.
%% Do not edit unless you really know what you are doing.
\documentclass[english,12pt]{article}
\usepackage[T1]{fontenc}
\usepackage[latin9]{inputenc}
\usepackage{amssymb}

\makeatletter

%%%%%%%%%%%%%%%%%%%%%%%%%%%%%% LyX specific LaTeX commands.
%% Binom macro for standard LaTeX users
\newcommand{\binom}[2]{{#1 \choose #2}}


%%%%%%%%%%%%%%%%%%%%%%%%%%%%%% User specified LaTeX commands.
\usepackage[margin=1in]{geometry}

\makeatother

\usepackage{babel}
\begin{document}

\title{Math 525: Assignment 2}

\date{\date{}}
\maketitle
\begin{enumerate}
\item (Independence) Let $A$ and $B$ be independent events. Show that
$A^{c}=\Omega\setminus A$ and $B^{c}=\Omega\setminus B$ are independent.
Also show that $A$ and $B^{c}$ are independent (and so too, by symmetry,
are $A^{c}$ and $B$).
\item (Conditional probability) Roll two (fair) dice. What is the probability
that at least one of the two dice is four given that their sum is
seven?
\item (Conditional probability) Let $(\Omega,\mathcal{F},\mathbb{P})$ be
a probability space. Let $B\in\mathcal{F}$ be an event with $\mathbb{P}(B)>0$.
Define the function $\mathbb{T}\colon\mathcal{F}\rightarrow[0,1]$
by $\mathbb{T}(A)=\mathbb{P}(A\mid B)$.
\begin{enumerate}
\item Show that $(\Omega,\mathcal{F},\mathbb{T})$ is a probability space.
\textbf{Hint}: since we already know that $\mathcal{F}$ is a $\sigma$-algebra
on $\Omega$, we need only check that $\mathbb{T}$ is a probability
measure.
\item Why did we require $\mathbb{P}(B)>0$?
\end{enumerate}
\item (Counting) Consider a circular table with $N\geq4$ seats. Barbie
and Ken are two of $N$ dinner guests to be seated at this table.
In how many ways can the dinner guests be seated such that...
\begin{enumerate}
\item Barbie and Ken are not adjacent.
\item Barbie and Ken are adjacent.
\end{enumerate}
\item (Counting) Prove the binomial theorem. That is, prove that for a positive
integer $n$ and real numbers $a$ and $b$,
\[
\left(a+b\right)^{n}=\sum_{k=0}^{n}\binom{n}{k}a^{k}b^{n-k}.
\]
\textbf{Hint}: an easy way to do this is with induction.
\item (Inverse images) Let $f\colon A\rightarrow B$. Show that for any
$H,J\subset B$,
\begin{enumerate}
\item $f^{-1}(H\cup J)=f^{-1}(H)\cup f^{-1}(J)$.
\item $f^{-1}(H^{c})=(f^{-1}(H))^{c}$.
\item $f^{-1}(H\cap J)=f^{-1}(H)\cap f^{-1}(J)$. \textbf{Hint}: use parts
(a) and (b).
\end{enumerate}
\item (Random variables) Let $(X_{n})_{n\geq1}$ be a sequence of random
variables. Prove that $\sup_{n}X_{n}$ is also a random variable.
\end{enumerate}

\end{document}
